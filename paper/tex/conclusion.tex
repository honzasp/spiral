\chapter{Závěr}

Představili jsme funkční implementaci netriviálního programovacího jazyka včetně
běhového prostředí a standardní knihovny. Podporovány jsou plnohodnotné funkce,
koncová volání i jednoduchý systém modulů.

Původní autorův záměr byl vytvořit překladač jazyka
Plch\footnote{\url{https://github.com/honzasp/plch}}, který měl být prakticky
stejný jako Spiral. Jeho první mezijazyk $^p\lambda_\chi$ byl velmi podobný
Spine, umožňoval však přímo pracovat s~celými čísly a funkcemi a měl proto i
jednoduchý typový systém. Druhý mezijazyk \texttt{tac} (\emph{\texttt{t}hree
\texttt{a}ddress \texttt{c}ode}), jehož zjednodušením vznikl Grit, se pak
překládal do assembleru za použití skutečného alokátoru registrů, generovaný kód
byl tedy poměrně kvalitní. Snaha o vytvoření tohoto překladače ale nakonec
selhala.

Rozvržení překladu přes jazyky Spine a Grit tak původně počítalo s~tím, že
většina optimalizací bude probíhat v~\uv{typově bezpečném} jazyku Spine a Grit
bude sloužit jen jako multiplatformní předstupeň assembleru, stejně jako jako
jeho neúspěšný předchůdce \texttt{tac}. Ukázalo se však, že optimalizovat
strukturovaný Spine je obtížnější než jednoduchý Grit, navíc některé důležité
aspekty (známá a neznámá volání) nelze v~bezpečném netypovaném jazyce vyjádřit
stejně přímo jako v~Gritu.

I když je jazyk Spine v~některých aspektech kvalitnějši než jisté široce
používané jazyky (například \uv{jazyk} PHP), trpí některými fundamentálními
nedostatky:

\begin{itemize}
  \item Autor je přesvědčen, že dynamicky typované jazyky dosáhly své popularity
    především díky absenci vysokoúrovňového praktického jazyka s~kvalitním
    statickým typovým systémem a bezpečnou správou paměti. Na tuto mezeru na
    trhu je zaměřen právě jazyk Rust, který ji dle autora může zaplnit velmi
    důstojně. Spiral je jako dynamický jazyk ve vývoji programovacích jazyků
    dnes již naprosto nezajímavá větev, jelikož se zde tísní spolu s~Pythonem,
    Luou, JavaScriptem a všemi ostatními deriváty Scheme a Lispu vůbec.

  \item V~jazyce chybí možnost pracovat se strukturami (záznamy, objekty,
    slovníky, tabulkami, ...), tedy n-ticemi, k~jejichž prvkům lze přistupovat
    pomocí symbolických jmen. Takovéto objekty buď vyžadují statické typování
    (\texttt{struct} v~C) nebo implementaci pomocí hešovací tabulky či jiné
    mapovací datové struktury (jako jsou objekty v~JavaScriptu, Ruby nebo
    Pythonu), což je však dle autora příliš velká cena za takto elementární
    operaci.
\end{itemize}

Kromě těchto vad, které jsou důsledkem návrhu jazyka, pak implementace trpí
množstvím dalších nedodělků, které by pro praktické použití musely být
napraveny:

\begin{itemize}
  \item Kvůli nízké kvalitě generátoru kódu, který ani není schopen umístit
    proměnné do registrů, a kvůli nutnosti volat externí funkce pro každou
    elementární operaci, včetně sčítání dvou celých čísel, jsou přeložené programy
    velmi pomalé. Neblahý vliv na výkon má i nutnost alokovat paměť pro každé
    použité reálné číslo. Pro nedostatek místa a času není součástí práce reálné
    srovnání rychlostí programu ve Spiral a programů v~jiných jazycích, ovšem
    výsledky by jistě byly pro Spiral velice nepříznivé. Nekvalitní garbage
    collector, který při každém úklidu paměti kopíruje všechny alokované
    objekty, a naivní (a tedy pomalé) algoritmy použité v~samotném překladači už
    jen katastrofickou situaci z~pohledu výkonu dovrší.

  \item Při parsování programu není nikde uchována informace o pozicích ve
    zdrojovém souboru, takže když nastane chyba při překladu nebo za běhu, není
    implementace schopna určit ani v~náznaku místo, kde chyba nastala, což činí
    vývoj jakéhokoli jen trochu většiho programu prakticky nemožným.

  \item Výše zmíněný generátor kódu je nejen děsivě neefektivní, ale zároveň
    velmi nepřehledný a špatně napsaný, a to i ve srovnání se zbytkem
    překladače.

  \item Standardní knihovna je pro praktické použití příliš minimalistická,
    chybí například funkce pro práci se souborovým systémem nebo základní datové
    struktury jako haldy nebo hešovací tabulky.

  \item Jazyk neumožňuje používat funkce s~proměnným počtem argumentů, takže
    standardní knihovna musí exportovat palety funkcí jako \texttt{tuple-0} až
    \texttt{tuple-8} nebo \texttt{str-cat-0} až \texttt{str-cat-8}.
    
  \item Zdrojový kód neobsahuje ani jeden komentář, jazyk Spiral dokonce syntaxi
    na zápis komentářů úplně postrádá.
\end{itemize}

Na závěr lze tedy říci, že jednoduchý programovací jazyk byl sice podle zadání
z~názvu práce přeložen, ovšem přínos práce končí tímto konstatováním. Byly
použity pouze běžné a v~literatuře mnohokrát popsané postupy, navíc často notně
zjednodušené.

