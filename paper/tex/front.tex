\begin{titlingpage}
  \sffamily
  \begin{flushright}
    \fontsize{50}{60}\selectfont
    překladač\\
    jednoduchého\\
    programovacího\\
    jazyka
  \end{flushright}
  ~\\[15.5cm]
  \begin{flushleft}
    \fontsize{20}{24}\selectfont
    jan špaček
  \end{flushleft}
\end{titlingpage}

\thispagestyle{empty}
\cleardoublepage

\begin{adjustwidth}{1.5cm}{1.5cm}

\section*{Abstract}

We present a~simple programming language Spiral and its implementation. Spiral
is an impure untyped functional language inspired by Scheme, featuring
first-class functions, tail-calls, module system and both mutable and immutable
data structures. We first translate the source language to Spine, a
continuation-passing style functional language. Spine is then transformed into
low-level imperative language Grit. Several optimization passes operate on this
intermediate language, including global value analysis and inlining. As the last
step of the compilation pipeline, we generate assembly, translate it with an
external assembler and link the resulting object file with the runtime library.
The runtime, written in \Cplusplus, defines core functions and objects and
manages the heap with a~moving garbage collector. The implementation includes a
basic standard library.

\section*{Abstrakt}

V~práci je představen jednoduchý programovací jazyk Spiral a jeho implementace.
Spiral je funkcionální jazyk inspirovaný Scheme, který zahrnuje funkce první
kategorie, koncová volání, systém modulů a měnitelné i neměnitelné datové
struktury. Zdrojový jazyk je nejprve přeložen do Spine, funkcionálního jazyka
založeného na CPS (continuation-passing style). Spine je poté transformován do
nízkoúrovňového imperativního jazyka Grit, na kterém operuje několik
optimalizačních fází, včetně globální optimalizace hodnot a inliningu.  Jako
poslední krok překladu je vygenerován externí assembler a přeložen externím
programem.  Výsledný objektový soubor je slinkován s~podpůrnou knihovnou.
Podpůrná běhová knihovna je napsaná v~\Cplusplus, definuje základní funkce a
objekty a spravuje paměť pomocí přemisťovacího garbage collectoru. Součástí
implementace je i základní standardní knihovna.

\end{adjustwidth}

\newpage

~\\[7cm]
\section*{Prohlášení}

Prohlašuji, že jsem svou práci vypracoval samostatně, použil jsem pouze
podklady (literaturu, SW atd.) uvedené v přiloženém seznamu a postup při
zpracování a dalším nakládání s prací je v souladu se zákonem č. 121/2000 Sb.,
o právu autorském, o právech souvisejících s právem autorským a o změně
některých zákonů (autorský zákon) v platném znění.\\[1.5cm]

V \makebox[6cm]{\dotfill} dne \makebox[6cm]{\dotfill}\\[4cm]

Podpis: \makebox[12cm]{\dotfill}

\newpage

\tableofcontents
