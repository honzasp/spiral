\chapter{Běhová knihovna}

Přeložený program potřebuje ke svému běhu podpůrnou běhovou knihovnu
(\emph{runtime}). V~této knihovně jsou definovány základní funkce, které využívá
standardní knihovna, a rovněž se stará o správu paměti.

Je špatnou programovací praxí používat nekonstantní globální proměnné, proto
jsou veškerá \uv{globální} data uložena v~objektu nazývaném pozadí
(\emph{background}, kvůli všudypřítomnosti zkráceno až na \emph{bg}). Odkaz na
tento objekt skrz funkce ve Spiral přenášen v~registru \texttt{\%edi}, v~běhové
knihovně pak explicitně jako první argument většiny funkcí.

\subsection{Reprezentace hodnot}

Všechny hodnoty jsou reprezentovány jako 32-bitové číslo. Hodnoty celých čísel
jsou reprezentovány přímo, a to vynásobené dvěmi, tudíž jejich nejnižší bit je
nulový. Takto je možno uložit $2^{31}$ čísel, takže rozsah celých čísel v~jazyku
Spiral je od $-2^{30}$ do $2^{30}-1$. Ostatní hodnoty jsou uloženy jako ukazatel
do paměti a jsou od celých čísel odlišeny tak, že jejich nejnižší bit je
nastaven na jedničku. Aby bylo možno rychle rozpoznat funkce, mají datové
objekty (tedy všechny, které nejsou funkce) po dělení čtyřmi zbytek 1, funkce
pak zbytek 3.  Původní ukazatel obdržíme snadno bitovou operací \emph{and}.

\begin{ttcode}
   <int>0 ... integer
  <ptr>01 ... data object
  <ptr>11 ... function
\end{ttcode}

Všechny objekty v~paměti mají v~prvních 4~bajtech uložen ukazatel na statickou
tabulku objektu (\emph{object table} neboli \emph{otable}). Pomocí této tabulky
lze dynamické objekty odlišit a pracovat s~nimi. Tabulka kromě ukazatelů na
funkce sloužící ke správě paměti obsahuje rovněž ukazatele na funkce pro převod
objektu na řetězec a pro porovnávání objektů. Tyto funkce jsou využity pro
implementaci funkcí \texttt{stringify}, \texttt{eqv?} a \texttt{equal?}
standardní knihovny, které musí být schopny pracovat se všemi typy objektů.

Obsah úseku paměti po ukazateli na tabulku objektu je již závislý na typu
objektu. Je zde například umístěn ukazatel na data a délka řetězce nebo prvky
n-tice. Funkce mají po tabulce objektu uloženu adresu začátku kódu funkce. Před
kódem je umístěna tabulka funkce (\emph{fun table} či zkráceně \emph{ftable}),
která obsahuje informace o počtu argumentů, slotů a zachycených hodnot, které
jsou potřeba pro korektní čtení zásobníku při úklidu paměti.

\subsection{Správa paměti}

Paměť objektů je strukturovaná do úseků (\emph{chunks}), které tvoří spojový
seznam. Když program požádá o paměť na uložení objektu, běhová knihovna použije
prostor prvního úseku. Pokud v~něm již není dostatek prostoru, může knihovna od
operačního systému získat další úsek a zařadit jej na začátek seznamu. Pokud
však program již alokoval velké množství paměti, je spuštěn garbage collector,
který všechny živé objekty zkopíruje do nových úseků paměti a staré úseky poté
uvolní k~dalšímu použití.

Jako živé jsou označeny všechny objekty, které jsou odkazovány z~proměnných
v~daném okamžiku běhu programu a všechny objekty, na které jiné živé objekty
odkazují. Nejprve jsou proto evakuovány (tedy zkopírovány ze staré oblasti do
nové a zachráněny před zničením) všechny objekty ze zásobníku. Poté jsou
evakuované objekty prohledány (\emph{scavenged} -- doslova \uv{hledat
v~odpadcích}) a všechny objekty, na které odkazují, jsou rovněž evakuovány. Aby
objekt nebyl evakuován vícekrát a nebyl pokaždé zkopírován, je na jeho místo po
evakuaci zapsán ukazatel (\emph{forward pointer}) odkazující na nové umístění
původního objektu. Zbylé objekty ze starých úseků paměti jsou zahozeny
(\emph{dropped}).

Některé objekty (například pole nebo řetězce) využívají pro uložení svého obsahu
další oblasti paměti mimo paměť objektů. Tyto objekty se o správu své paměti
starají samy a když jsou zahozeny, svou paměť uvolní.
