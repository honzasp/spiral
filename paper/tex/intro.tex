\chapter{Úvod}

Procesory počítačů pracují se sadou jednoduchých instrukcí, které lze rychle a
efektivně vykonávat. Programování v~těchto základních stavebních jednotkách je
ovšem natolik pomalé, obtížné a náchylné k~chybám, že se rychle po sestrojení
prvních počítačů vyvinuly rovněž první programovací jazyky, které umožňují
programy vytvářet v~podobě, která je pro člověka jednodušší. Vývoj v~takovém
programovacím jazyku je nesrovnatelně efektivnější než v~kódu počítače a
programy v~něm napsané je často jednodušší použít i na jiných počítačích, než na
které byly původně určeny.

Aby program napsaný ve vyšším programovacím jazyku mohl být spuštěn, musí pro
tento jazyk existovat interpret nebo kompilátor (překladač). Interpret je
program, který přečte zdrojový kód interpretovaného jazyka a sám vykonává
operace v~něm obsažené. Překladač naproti tomu přeloží program v~programovacím
jazyku do strojového kódu, takže je poté tento program možno spouštět přímo na
procesoru počítače, k~čemuž již není potřeba původní zdrojový kód ani překladač.
Vytvořit interpret je většinou poměrně jednoduché, ovšem interpretovaný program
zpravidla dosahuje horšího výkonu než program přeložený přímo  do strojového
kódu.

Programovací jazyky se většinou soustředily na organizaci kódu a průběhu
výpočtu, ovšem menší pozornost byla věnována problému organizace paměti. Dnes
proto existují v~zásadě dva typy jazyků: nízkoúrovňové jazyky jako C a
\Cplusplus, kde je správa paměti plně v~režii programátora, a všechny ostatní
jazyky, kde je paměť spravována pomocí garbage collectoru, který paměť uvolňuje
automaticky bez zásahu programátora. Oba tyto přístupy však mají své problémy.
Ruční správa paměti je extrémně náchylná k~chybám, které je obtížné odhalit, a
navíc často představují bezpečnostní riziko (asi nejznámější bezpečnostní chybou
tohoto typu byl Heartbleed \cite{heartbleed}). Na druhou stranu garbage
collectory zbavují programátora možnosti nakládat volně s~pamětí a ubírají
programům na výkonu.

\section{Struktura překladače}

Struktura běžného překladače vypadá přibližně takto:
\cite{grune2012modern,appel1998modern}:

\begin{itemize} 
  \item Lexikální a syntaktická analýza zdrojového jazyka, jejímž výsledkem je
    syntaktický strom.
  \item Sémantická analýza zdrojového jazyka (spojení jmen a jejich definic,
    kontrola typů). Tato část překladače je spolu s~lexikální a syntaktickou
    analýzou označována jako front-end.
  \item Generování přechodného jazyka (\emph{intermediate language}), který je
    jednodušší a pro další zpracování vhodnější než zdrojový jazyk.
  \item Analýza a optimalizace přechodného jazyka.
  \item Překlad přechodného jazyka do assembleru nebo objektového souboru
    (back-end).
\end{itemize}

Jako příklad lze uvést překladač GHC \cite{jones1993glasgow,haskellreport2010},
který nejprve zdrojový jazyk Haskell přeloží do jazyka Core
\cite{jones1996compiling}, nad kterým probíhají optimalizace
\cite{jones1997optimiser,santos1995compilation}. Core je poté přeložen do jazyka
STG \cite{jones1992implementing}, ze kterého je následně generován kód.

Často je možné několik zdrojových jazyků překládat do jednoho přechodného jazyka
a sdílet tak společný back-end. Jedním z~takových projektů je LLVM \cite{llvm},
na kterém je založen například překladač Clang \cite{clang} jazyků C a
\Cplusplus{} nebo překladač jazyka Rust \cite{rust}.
