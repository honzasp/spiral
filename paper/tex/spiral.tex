\chapter{Popis jazyka Spiral}

Samotný jazyk je velmi prostý, jelikož poskytuje pouze
nástroje pro definici modulů, tvorbu funkcí a proměnných, podmíněné
vyhodnocování a literály.  Primitivní operace (například sčítání čísel či
přístup k~prvku pole) jsou definovány ve standardní knihovně, kde jsou
implementovány jako externí volání podpůrné knihovny. Za zmínku stojí, že
identifikátory jako \texttt{+} a \texttt{*} nejsou zabudované operátory, ale
klasické proměnné. 

Proměnné jsou neměnitelné, neexistuje proto žádný příkaz přiřazení, některé
objekty ze standardní knihovny, například pole, je ale možno interně upravovat.
Funkce, které mění nějaký objekt, je zvykem pojmenovávat s~vykřičníkem na konci
(\texttt{array-set!}).

Volání funkcí v~koncových pozicích (\emph{tail-calls}) jsou implementována tak,
že umožňují neomezené množství rekurzivních volání bez hrozby přetečení
zásobníku, což umožňuje vyjádřit všechny iterace pomocí rekurze.

\section{Gramatika}

Syntaxe jazyka Spiral je založena na syntaxi jazyka Scheme \cite{scheme7}.
Program je tak zapsán ve formě uzávorkovaných prefixových výrazů
(\emph{s-expressions}). Tento způsob zápisu je jednoduchý na parsování a
příjemný na psaní.

\subsection{Programy a moduly}
\begin{ttcode}
<top-level>   = <program> | <module>
<program>     = (program <stmt>...)
<module>      = (module <ident> <decl>...)
<decl>        = (export <ident>...) | <stmt>
\end{ttcode}

Program obsahuje seznam příkazů (\emph{statement}, \texttt{<stmt>}), které jsou
postupně vykonány. Modul obsahuje deklarace, které kromě příkazů zahrnují i
exporty, které modul poskutuje ostatním modulům a hlavnímu programu.

~
\subsection{Příkazy}
\begin{ttcode}
<import-def>  = <ident>
              | (only <import-def> <ident>...)
              | (except <import-def> <ident>...)
              | (prefix <import-def> <ident>)
<stmt>        = (import <import-def>...)
              | (fun <ident> (<ident>...) <stmt>...)
              | (var <ident> <expr>)
              | <expr>
\end{ttcode}

\begin{description}

\item[\texttt{(import <import-def>...)}] Příkaz importu načte modul a umožní
  používat jeho exportované proměnné v~aktuálním kontextu. Je možno importovat jen
  vybraná jména pomocí \texttt{(only ...)}, nebo importovat všechny kromě zadaných
  \texttt{(except ...)} nebo všem importovaným jménům předřadit prefix
  \texttt{(prefix ...)}.

  Příklady importů:

  \begin{itemize}
    \item \texttt{(import std std.math)} importuje všechna jména z~modulů
      \texttt{std} a \texttt{std.math}.
    \item \texttt{(import (only std + - *))} importuje ze \texttt{std} pouze jména
      \texttt{+}, \texttt{-} a \texttt{*}.
    \item \texttt{(import (except std.math sin cos))} importuje
      z~\texttt{std.math} všechna jména kromě \texttt{sin} a \texttt{cos}.
    \item \texttt{(import (prefix (only std.math div mod) m.))} importuje z~modulu
      \texttt{std.math} jména \texttt{div} a \texttt{mod} a přidá jim prefix
      \texttt{m.} (takže se na ně bude možno odkázat jako \texttt{m.div} a
      \texttt{m.mod}).
  \end{itemize}

\item[\texttt{(fun <fun-name> (<arg>...) <body-stmt>...)}] definuje pojmenovanou
  funkci. Funkce, které jsou takto definovány těsně za sebou, mohou být vzájemně
  rekurzivní, takže funkce definovaná v~pořadí dříve \uv{vidí} definici pozdější
  funkce (včetně sebe sama). Kromě toho je ve funkcích možno použít veškeré
  proměnné definované vně (taková funkce se nazývá \emph{closure}).

\item[\texttt{(var <var-name> <value>)}] definuje proměnnou. Proměnná, narozdíl
  of funkce,odkazovat na sebe samu nemůže, jelikož její hodnota může být použíta
  až po její definici.

\item[\texttt{<expr>}] -- pokud je na místě příkazu nalezen výraz, je vyhodnocen.
  Pokud je na posledním místě v~posloupnosti příkazů, je jeho hodnota použíta jako
  hodnota celé posloupnosti, jinak je výsledek zahozen.

\end{description}

\subsection{Výrazy}

\begin{gram}
<expr>        = (if <expr> <expr> <expr>)
              | (cond (<expr> <stmt>...)...)
              | (when <expr> <stmt>...)
              | (unless <expr> <stmt>...)
              | (do ((<ident> <expr> <expr>)...) (<expr> <stmt>...) <stmt>...)
              | (and <expr>...)
              | (or <expr>...)
              | (begin <stmt>...)
              | (let ((<ident> <expr>)...) <stmt>...)
              | (lambda (<ident>...) <stmt>...)
              | (<expr> <expr>...)
              | (extern <ident> <expr>...)
              | <integer-literal>
              | <float-literal>
              | <string-literal>
\end{gram}

\begin{description}
  \item[\texttt{<integer-literal>}, \texttt{<float-literal>}] jsou číselné
    konstanty, které se vyhodnotí na číslo, které reprezentují. Číslice je
    množno oddělovat podtržitky a v~zápisu desetinného čísla je možno použít
    exponent.

  \item[\texttt{<string-literal>}] je řetězcový literál uzavřený v~uvozovkách,
    ve kterém je možné použít escape sekvence z~jazyka C a podobných
    (například \texttt{\textbackslash{}n} je znak nového řádku,
    \texttt{\textbackslash{}"} je uvozovka).

  \item[\texttt{(if <condition> <then> <else>)}] je podmíněný příkaz, který
    nejprve vyhodnotí podmínku, a pokud je pravdivá, vyhodnotí první výraz,
    jinak vyhodnotí druhý výraz. Za pravdivé jsou považovány všechny hodnoty
    kromě \texttt{false}.

  \item[\texttt{(when <condition> <body-stmt>...)}] vyhodnotí podmínku a pokud
    je pravdivá, vyhodnotí následující příkazy. 

  \item[\texttt{(unless <condition> <body-stmt>...)}] je obdobou \texttt{when},
    ale příkazy vyhodnotí pokud je podmínka nepravdivá.

  \item[\texttt{(cond (<condition> <stmt>...)..)}] postupně zkouší vyhodnocovat
    podmínky a vyhodnotí příkazy u první, která je pravdivá.

  \item[\texttt{(and <expr>...)}] vyhodnocuje výrazy zleva a vrátí hodnotu
    prvního, který je nepravdivý, čímž implementuje logické \uv{a}.

  \item[\texttt{(or <expr>...)}] vyhodnocuje výrazy zleva a vrátí hodnotu
    prvního, který je pravdivý, což je obdoba logického \uv{nebo}.

  \item[\texttt{(let ((<var> <expr>)...) <body-stmt>...)}] přiřadí proměnným
    hodnoty daných výrazů a poté vyhodnotí následující příkazy.

  \item[\texttt{(lambda (<arg>...) <body-stmt>...)}] vytvoří anonymní funkci
    s~danými argumenty a příkazy, které tvoří její tělo. Uvnitř funkce je možno
    používat proměnné z~vnějšku funkce.

  \item[\texttt{(<fun> <arg>...)}] je výraz volání. Nejprve se vyhodnotí
    všechny argumenty zleva doprava a poté výraz funkce, která je vzápětí
    zavolána. Pokud se první výraz nevyhodnotí na funkci, program skončí s~chybou.

  \item[\texttt{(extern <fun-name> <arg>...)}] je volání externí funkce
    (definované v~jazyku C) s~danými argumenty. Normálně se tato volání objevují
    pouze ve standardní knihovně. Počet argumentů ani existenci funkce není
    možno zkontrolovat, korektnost je tedy na programátorovi.

  \item[\texttt{(begin <body-stmt>...)}] vyhodnotí všechny příkazy a vrátí
    hodnotu posledního.

  \item[\texttt{(do ((<var> <init> <next>)...) (<exit-condition> <exit-stmt>...)
    <body-stmt>...)}] je příkaz cyklu. Do všech proměnných je neprve přiřazena
    počáteční hodnota (\texttt{<init>}). Poté je vyhodnocena podmínka
    (\texttt{<edit-condition>}) a pokud je pravdivá, vyhodnotí se následující
    příkazy (\texttt{<exit-stmt>...}) a cyklus se ukončí. V~opačném případě jsou
    vyhodnoceny příkazy v~těle cyklu (\texttt{<body-stmt>...}), poté jsou do
    proměnných přiřazeny hodnoty \texttt{<next>} a cyklus se opakuje novou
    kontrolou podmínky.

    Například následující program spočte sto čísel Fibonacciho posloupnosti,
    přičemž každé vypíše, a na konci vypíše \texttt{done}:

\begin{spiral}
(program
  (import std)
  (do ((f1 0 f1)
       (f2 1 (+ f1 f2))
       (i  1 (+ i 1)))
    ((>= i 100)
      (println "done"))
    (println f1)))
\end{spiral}
\end{description}

\section{Standardní knihovna a základní typy}

Standardní knihovna je implementována v~modulu \texttt{std} a definuje základní
funkce. Jelikož program ve Spiral nemá implicitně definovaná žádná jména, je
použití standardní knihovny prakticky nutností.

\section{Ovládání překladače z~příkazové řádky}

Překladač je možno ovládat pomocí argumentů na příkazové řádce.

\section{Příklady}

Na závěr uvádíme několik drobných programů napsaných v~jazyce Spiral.
