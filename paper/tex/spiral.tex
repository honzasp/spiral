\chapter{Popis jazyka Spiral}

Samotný jazyk je velmi prostý, jelikož poskytuje pouze nástroje pro definici
modulů, tvorbu funkcí a proměnných, základní větvení a literály.  Primitivní
operace (například sčítání čísel či přístup k~prvkům pole) jsou definovány ve
standardní knihovně, kde jsou implementovány jako externí volání podpůrné
knihovny. Za zmínku stojí, že identifikátory jako \texttt{+} a \texttt{*} nejsou
zabudované operátory, ale klasické proměnné. 

Proměnné jsou neměnitelné, neexistuje proto žádný příkaz přiřazení. Některé
objekty ze standardní knihovny, například pole, je ale možno interně upravovat.
Funkce, které mění nějaký objekt, je zvykem pojmenovávat s~vykřičníkem na konci
(\texttt{array-set!}).

Volání funkcí v~koncových pozicích (\emph{tail-calls}) jsou implementována tak,
že umožňují neomezené množství rekurzivních volání bez hrozby přetečení
zásobníku, což umožňuje efektivně vyjádřit všechny iterace pomocí rekurze.

Funkce jsou hodnoty podobně jako čísla nebo řetězce, je s~nimi tedy možno bez
omezení pracovat jako v~jiných funkcionálních jazycích. Ve funkci je možno
použít proměnné definované vně těla funkce. Tyto proměnné se nazývají zachycené
(\emph{captured}) a jejich hodnoty jsou spolu s~adresou funkce uloženy v~paměti
v~objektu, který tuto funkci reprezentuje (\emph{closure}).

\section{Gramatika}

Syntaxe jazyka Spiral je založena na syntaxi jazyka Scheme \cite{scheme7}.
Program je tak zapsán ve formě uzávorkovaných prefixových výrazů
(\emph{s-expressions}). Tento způsob zápisu je jednoduchý na parsování a
příjemný na psaní.

\subsection{Programy a moduly}
\begin{ttcode}
<top-level>   = <program> | <module>
<program>     = (program <stmt>...)
<module>      = (module <ident> <decl>...)
<decl>        = (export <ident>...) | <stmt>
\end{ttcode}

Program obsahuje seznam příkazů (\emph{statements}, \texttt{<stmt>...}), které
jsou postupně vykonány. Modul obsahuje deklarace, které kromě příkazů zahrnují i
exporty, které modul poskutuje ostatním modulům a hlavnímu programu.

~
\subsection{Příkazy}

Příkazy slouží k~definici importovaných jmen, pojmenovaných funkcí i obyčejných
proměnných.

\begin{ttcode}
<stmt>        = (import <import-def>...)
              | (fun <ident> (<ident>...) <stmt>...)
              | (var <ident> <expr>)
              | <expr>
<import-def>  = <ident>
              | (only <import-def> <ident>...)
              | (except <import-def> <ident>...)
              | (prefix <import-def> <ident>)
\end{ttcode}

\begin{description}

\item[\texttt{(import <import-def>...)}] Příkaz importu načte modul a umožní
  používat jeho exportované proměnné v~aktuálním kontextu. Je možno importovat
  vybraná jména pomocí \texttt{(only ...)}, nebo importovat všechny kromě zadaných
  \texttt{(except ...)} nebo všem importovaným jménům předřadit prefix
  \texttt{(prefix ...)}.

  Příklady importů:

  \begin{itemize}
    \item \texttt{(import std std.math)} importuje všechna jména z~modulů
      \texttt{std} a \texttt{std.math}.
    \item \texttt{(import (only std + - *))} importuje ze \texttt{std} pouze jména
      \texttt{+}, \texttt{-} a \texttt{*}.
    \item \texttt{(import (except std.math sin cos))} importuje
      z~\texttt{std.math} všechna jména kromě \texttt{sin} a \texttt{cos}.
    \item \texttt{(import (prefix (only std.math div mod) m.))} importuje z~modulu
      \texttt{std.math} jména \texttt{div} a \texttt{mod} a přidá jim prefix
      \texttt{m.} (takže se na ně bude možno odkazovat jako \texttt{m.div} a
      \texttt{m.mod}).
  \end{itemize}

\item[\texttt{(fun <fun-name> (<arg>...) <body-stmt>...)}] definuje pojmenovanou
  funkci. Funkce, které jsou takto definovány těsně za sebou, mohou být vzájemně
  rekurzivní, jelikož funkce definovaná v~pořadí dříve \uv{vidí} i definice
  pozdějších funkcí (včetně sebe samotné). Kromě toho je ve funkcích možno použít
  veškeré předem definované proměnné.

\item[\texttt{(var <var-name> <value>)}] definuje proměnnou. Proměnná, narozdíl
  od funkce, odkazovat na sebe samu nemůže, jelikož její hodnota může být použita
  až po její definici.

\item[\texttt{<expr>}] -- pokud je na místě příkazu nalezen výraz, je vyhodnocen.
  Pokud je na posledním místě v~posloupnosti příkazů, je jeho hodnota použita jako
  hodnota celé posloupnosti, jinak je výsledek zahozen.

\end{description}

\subsection{Výrazy}

Všechny zbývající konstrukce jazyka jsou výrazy (\emph{expressions}), což
znamená, že vrací hodnotu.  Pokud není žádná rozumná hodnota, kterou by měl
výraz vrátit (například prázdný výraz \texttt{(begin)}), vrátí výraz hodnotu
\texttt{false}.

\begin{gram}
<expr>        = (if <expr> <expr> <expr>)
              | (cond (<expr> <stmt>...)...)
              | (when <expr> <stmt>...)
              | (unless <expr> <stmt>...)
              | (do ((<ident> <expr> <expr>)...) (<expr> <stmt>...) <stmt>...)
              | (and <expr>...)
              | (or <expr>...)
              | (begin <stmt>...)
              | (let ((<ident> <expr>)...) <stmt>...)
              | (lambda (<ident>...) <stmt>...)
              | (<expr> <expr>...)
              | (extern <ident> <expr>...)
              | <integer-literal>
              | <float-literal>
              | <string-literal>
              | <character-literal>
\end{gram}

\begin{description}
  \item[\texttt{<integer-literal>}, \texttt{<float-literal>}] jsou číselné
    konstanty, které se vyhodnotí na číslo, které reprezentují. Číslice je
    množno oddělovat podtržítky a v~zápisu desetinného čísla je možno použít
    exponent.

  \item[\texttt{<character-literal>}] je znaková konstanta, uzavřená
    v~jednoduchách uvozovkách, která se vyhodnotí na číselnou hodnotu znaku.

  \item[\texttt{<string-literal>}] je řetězcový literál uzavřený v~uvozovkách,
    ve kterém je možné použít escape sekvence z~jazyka C (například
    \texttt{\textbackslash{}n} je znak nového řádku, \texttt{\textbackslash{}"}
    je uvozovka).

  \item[\texttt{(if <condition> <then> <else>)}] je podmíněný příkaz, který
    nejprve vyhodnotí podmínku, a pokud je pravdivá, vyhodnotí první výraz,
    jinak vyhodnotí druhý výraz. Za pravdivé jsou považovány všechny hodnoty
    kromě \texttt{false}.

  \item[\texttt{(when <condition> <body-stmt>...)}] vyhodnotí podmínku a pokud
    je pravdivá, vyhodnotí následující příkazy. 

  \item[\texttt{(unless <condition> <body-stmt>...)}] je obdobou \texttt{when},
    ale příkazy vyhodnotí pouze pokud je podmínka nepravdivá.

  \item[\texttt{(cond (<condition> <stmt>...)..)}] postupně zkouší vyhodnocovat
    podmínky a vyhodnotí příkazy u první, která je pravdivá.

  \item[\texttt{(and <expr>...)}] vyhodnocuje výrazy zleva a vrátí hodnotu
    prvního, který je nepravdivý, čímž implementuje logické \uv{a}.

  \item[\texttt{(or <expr>...)}] vyhodnocuje výrazy zleva a vrátí hodnotu
    prvního, který je pravdivý, což je obdoba logického \uv{nebo}.

  \item[\texttt{(let ((<var> <expr>)...) <body-stmt>...)}] přiřadí proměnným
    hodnoty daných výrazů a poté vyhodnotí následující příkazy.

  \item[\texttt{(lambda (<arg>...) <body-stmt>...)}] vytvoří anonymní funkci
    s~danými argumenty a příkazy, které tvoří její tělo. Uvnitř funkce je opět
    možno používat proměnné z~vnějšku funkce.

  \item[\texttt{(<fun> <arg>...)}] je zápis volání. Nejprve se vyhodnotí všechny
    argumenty zleva doprava a poté funkce, která je poté zavolána. Pokud se
    první výraz nevyhodnotí na funkci, program skončí s~chybou.

  \item[\texttt{(extern <fun-name> <arg>...)}] je volání externí funkce
    (definované v~jazyku C) s~danými argumenty. Obvykle se tato volání objevují
    pouze ve standardní knihovně. Počet argumentů ani existenci funkce není
    možno zkontrolovat, korektnost je tedy závislá na programátorovi.

  \item[\texttt{(begin <body-stmt>...)}] vyhodnotí všechny příkazy a vrátí
    hodnotu posledního.

  \item[\texttt{(do ((<var> <init> <next>)...) (<exit-condition> <exit-stmt>...)
    <body-stmt>...)}] je výraz cyklu. Do všech proměnných je neprve přiřazena
    počáteční hodnota (\texttt{<init>}). Poté je vyhodnocena podmínka
    (\texttt{<exit-condition>}) a pokud je pravdivá, vyhodnotí se následující
    příkazy (\texttt{<exit-stmt>...}) a cyklus se ukončí. V~opačném případě jsou
    vyhodnoceny příkazy v~těle cyklu (\texttt{<body-stmt>...}), poté jsou do
    proměnných přiřazeny hodnoty \texttt{<next>} a cyklus se opakuje novou
    kontrolou podmínky.

    Například následující program spočte sto čísel Fibonacciho posloupnosti,
    přičemž každé vypíše, a na konci vypíše \texttt{done}:

\begin{spiral}
(program
  (import std)
  (do ((f1 0 f1)
       (f2 1 (+ f1 f2))
       (i  1 (+ i 1)))
    ((>= i 100)
      (println "done"))
    (println f1)))
\end{spiral}
\end{description}

\section{Standardní knihovna a základní typy}

Standardní knihovna definuje základní funkce. Jelikož program ve Spiral nemá
implicitně definovaná žádná jména, je import standardní knihovny prakticky
nutností.

\begin{description}
  \item[\texttt{std.core}] definuje základní operace s~čísly (\texttt{+},
    \texttt{mod}, \texttt{<}, \texttt{==}, ...), predikáty ekvivalence
    (\texttt{eqv?}, \texttt{equal?}) a základní funkci pro výstup
    (\texttt{println}).  Rovněž definuje logické proměnné \texttt{true} a
    \texttt{false} (jako hodnoty výrazů \texttt{(and)} a \texttt{(or)}).

  \item[\texttt{std.array}] poskytuje funkce pro práci s~poli
    (\texttt{array-new}, \texttt{array-push!}, \texttt{array-get},
    \texttt{array-empty?}, ...). Pole jsou indexována celými čísly od nuly a
    jejich délka i obsah jsou měnitelné.

  \item[\texttt{std.tuple}] umožňuje používat n-tice (\emph{tuples}), a to
    v~délce od 0 do 8 prvků. Modul definuje konstruktory (\texttt{tuple-2},
    ...), funkce pro přístup k~jednotlivým prvkům (\texttt{get-0},
    \texttt{get-2}) a predikáty (\texttt{tuple?}, \texttt{tuple-2?},
    \texttt{tuple-n?}, ...). Prvky n-tic jsou neměnitelné.

  \item[\texttt{std.cons}] definuje funkce pro práci s~páry (\emph{cons}), které
    se převážně používají ve formě spojových seznamů, kdy první prvek páru
    (\emph{car}) ukládá hodnotu prvního prvku seznamu a druhý prvek (\emph{cdr})
    odkazuje na zbytek seznamu. Konec seznamu reprezentuje speciální hodnota
    (\emph{null} či \emph{nil}), pro kterou je ve Spiral použito \texttt{false}.
    Tento modul definuje základní funkce pro práci s~páry (\texttt{cons},
    \texttt{car}, \texttt{cdr}, \texttt{cons?}, ...) i pro manipulaci se seznamy
    (\texttt{list?}, \texttt{list-len}, \texttt{list-append},
    \texttt{list-reverse}, ...). Páry jsou neměnné, což znamená, že není možno
    vytvořit kruhový seznam.

  \item[\texttt{std.string}] exportuje funkce pracující s~řetězci
    (\texttt{str-len}, \texttt{str-get}, \texttt{stringify}, \texttt{str-cat-2},
    \texttt{str-cat-3}, ...). Řetězce jsou neměnitelné a složené z~bytů.

  \item[\texttt{std.io}] slouží pro vstupní a výstupní operace
    (\emph{input/output}) se soubory a standardními proudy
    (\texttt{io-file-open}, \texttt{io-close}, \texttt{io-stdin},
    \texttt{io-write-line}, \texttt{io-read-line}, \texttt{io-read-number},
    ...).

  \item[\texttt{std.env}] poskytuje programu přístup k~argumentům z~příkazové
    řádky (\texttt{env-get-argv}) a k~proměnným prostředí (\texttt{env-get-var}).

  \item[\texttt{std.math}] implementuje základní matematické funkce jako
    \texttt{abs}, \texttt{neg}, \texttt{sin} nebo \texttt{atan-2}.

  \item[\texttt{std.test}] je minimalistická knihovna pro tvorbu jednotkových
    testů.
\end{description}

Modul \texttt{std} reexportuje základní definice (například \texttt{+} nebo
\texttt{car}) a je poskytován pro pohodlí programátora.

\section{Ovládání překladače z~příkazové řádky}

Překladač je možno ovládat pomocí argumentů na příkazové řádce:

\begin{description}
  \item[\texttt{-o}, \texttt{-{}-output}] umožňuje určit výstupní soubor.
    Výstup je jinak umístěn do stejného adresáře jako vstupní soubor se stejným
    jménem a příponou určenou podle typu výstupu.
  \item[\texttt{-I}, \texttt{-{}-include}] přidá adresář, ve kterém se hledají
    \texttt{import}ované moduly.
  \item[\texttt{-e}, \texttt{-{}-emit}] nastaví typ výstupu: \texttt{sexpr} přečte
    vstupní s-výraz a vypíše jej ve formátované podobě, \texttt{spiral} vypíše
    syntaktický strom Spiral v~interní podobě, \texttt{spine} a
    \texttt{grit} vypíšou program v~odpovídajícím mezijazyku jako s-výraz,
    \texttt{asm} vypíše vygenerovaný assembler v~interní podobě, \texttt{gas}
    vypíše assembler jako vstup pro GNU Assembler a \texttt{exec} program
    přeloží a slinkuje s~běhovou knihovnou (což je výchozí chování).
  \item[\texttt{-t}, \texttt{-{}-runtime}] určí soubor s~běhovou knihovnou, se
    kterou bude program slinkován.
  \item[\texttt{-{}-link-cmd}] nastaví linkovací program. Výchozí je \texttt{clang},
    který sám zavolá systémový linker tak, aby program správně slinkoval se
    standardní knihovnou jazyka C (\texttt{libc}).
  \item[\texttt{-{}-gas-cmd}] umožňuje změnit assembler místo výchozího
    \texttt{as}.
  \item[\texttt{-O}, \texttt{-{}-opt-level}] umožňuje nastavit úroveň optimalizace
    od 0 do 3.
\end{description}

\section{Příklady}

Na závěr uvádíme několik drobných programů napsaných v~jazyce Spiral.

\subsection{Počítání prvočísel}

Následující program vypíše prvních tisíc prvočísel. Funkce
\texttt{make-prime-gen} vytvoří generátor, tedy funkci, která při každém dalším
zavolání vrátí další prvočíslo. Tato funkce má interně uloženo pole doposud
nalezených prvočísel.

\begin{spiral}
(program
  (import std)
  (import std.array)
  (import std.test)

  (fun make-prime-gen ()
    (var primes (array-new))
    (lambda ()
      (fun find-next-prime (x)
        (fun check-prime (i)
          (var p (array-get primes i))
          (cond
            ((> (* p p) x) true)
            ((== (mod x p) 0) false)
            (true (check-prime (+ 1 i)))))
        (if (check-prime 1) x (find-next-prime (+ 2 x))))
      (cond
        ((== (array-len primes) 0)
          (array-push! primes 2) 2)
        ((== (array-len primes) 1)
          (array-push! primes 3) 3)
        (true 
          (var next-prime (find-next-prime (+ 2 (array-last primes))))
          (array-push! primes next-prime)
          next-prime))))

  (var gen (make-prime-gen))
  (do ((i 0 (+ i 1)))
    ((>= i 1000))
    (println (gen))))
\end{spiral}

\subsection{Počítání celých mocnin a odmocnin}

Tento program aproximuje $\sqrt[n]{x}$ Newtonovou metodou (funkce \texttt{root})
a počítá $x^n$ pomocí binárního umocňování (funkce \texttt{power}). Oba
algoritmy jsou poté otestovány pomocí knihovny \texttt{std.test}.

\begin{spiral}
(program
  (import std)
  (import std.math)
  (import std.test)

  (var min-del 0.000001)
  (fun root (n a)
    (fun iter (x)
      (var b (/ a (power (- n 1) x)))
      (var x-del (/ (- b x) n))
      (if (< (abs x-del) min-del) x (iter (+ x x-del))))
    (iter (* a 1.0)))
  (fun power (n a)
    (cond
      ((== n 0) 1)
      ((== n 1) a)
      ((== (mod n 2) 0) (square (power (div n 2) a)))
      (true (* a (square (power (div n 2) a))))))
  (fun square (a) (* a a))

  (var eps 0.00001)
  (test "powers" (lambda (t) 
    (assert-near-eq t eps (power 2 10) 100)
    (assert-near-eq t eps (power 3 2) 8)
    (assert-near-eq t eps (power 1 33) 33)))
  (test "roots" (lambda (t)
    (assert-near-eq t eps (root 3 1000) 10)
    (assert-near-eq t eps (root 2 49) 7)
    (assert-near-eq t eps (root 4 256) 4)))
  (test "powered roots" (lambda (t)
    (assert-near-eq t eps (power 2 (root 2 2)) 2)
    (assert-near-eq t eps (power 3 (root 3 100)) 100)
    (assert-near-eq t eps (power 2 (root 2 13)) 13)
    (assert-near-eq t 0.01 (power 5 (root 5 12345)) 12345))))
\end{spiral}
