\chapter{Implementace}

Celá implementace je dostupná v~gitovém repozitáři ze serveru GitHub na adrese
\url{https://github.com/honzasp/spiral}. V~adresáři \texttt{src/} se nachází
zdrojové kódy samotného překladače, adresář \texttt{rt/} obsahuje implementaci
běhové knihovny, v~adresář \texttt{stdlib/} jsou definovány moduly standardní
knihovny jazyka Spiral a v~adresáři \texttt{tests/} jsou umístěny celkové testy.

\section{Překladač}

Překladač je napsaný v~jazyce Rust \cite{rust} a s~více než 6000 řádky
kódu\footnote{Započítány jsou pouze řádky, které obsahují alespoň dva neprázdné
znaky.} (včetně jednotkových testů) se jedná o největší část implementace. Celý
program je rozdělen do modulů a toto rozdělení úzce souvisí s~rozdělením postupu
překladu na jednotlivé mezijazyky.

\begin{description}
  \item[\texttt{main}] obsahuje hlavní funkci, která podle předaných argumentů
    přeloží a slinkuje předaný vstupní soubor s~programem.
  \item[\texttt{args}] zodpovídá za dekódování argumentů z~příkazové řádky.

  \item[\texttt{sexpr::syntax}] definuje datové typy s-výrazů.
  \item[\texttt{sexpr::parse}] poskytuje parser s-výrazů.
  \item[\texttt{sexpr::pretty_print}] převádí s-výrazy z~interní podoby opět do
    textového formátu v~lidsky čitelné podobě.
  \item[\texttt{sexpr::to_spiral}] umožňuje z~s-výrazu získat syntaktický strom
    jazyka Spiral.
  \item[\texttt{sexpr::to_spine}] z~s-výrazu načte definici programu v~jazyku
    Spine (což se využívá pro testování).
  \item[\texttt{sexpr::to_grit}] přečte z~s-výrazu definici programu v~jazyku
    Grit (opět využito pro testování).

  \item[\texttt{spiral::syntax}] definuje typy pro syntaktický strom jazyka
    Spiral.
  \item[\texttt{spiral::env}] poskytuje pomocný objekt prostředí
    (\emph{environment}) pro průchod syntaktického stromu.
  \item[\texttt{spiral::imported}] slouží pro získání seznamu všech
    importovaných modulů z~jednoho modulu či programu.
  \item[\texttt{spiral::to_spine}] překládá program ze Spiral do Spine, k~čemuž
    využívá:
  \item[\texttt{spiral::to_spine::mods}] k~překladu celých modulů,
  \item[\texttt{spiral::to_spine::decls}] k~překladu deklarací,
  \item[\texttt{spiral::to_spine::stmts}] k~překladů příkazů a
  \item[\texttt{spiral::to_spine::exprs}] k~překladu výrazů.

  \item[\texttt{spine::syntax}] definuje syntaktický strom jazyka Spine.
  \item[\texttt{spine::env}] poskytuje prostředí pro průchod tímto stromem.
  \item[\texttt{spine::check}] implementuje kontrolu korektnosti
    (\emph{well-formedness}), která je využívána pro testování.
  \item[\texttt{spine::eval}] umožňuje vyhodnotit programy v~jazyku Spine, čehož
    se opět využívá při testování.
  \item[\texttt{spine::free}] umožňuje zjistit všechny volné proměnné v~definici
    funkce, čehož se využívá při zjištění všech zachycených proměnných.
  \item[\texttt{spine::onion}] poskytuje typ \uv{cibule} pro překlad ze Spiral.
  \item[\texttt{spine::to_grit}] překládá program ze Spine do jazyka Grit.
  \item[\texttt{spine::to_sexpr}] převede Spine na s-výraz.

  \item[\texttt{grit::syntax}] definuje syntaxi jazyka Grit.
  \item[\texttt{grit::optimize_dead_vals}] implementuje optimalizaci nevyužitých
    hodnot.
  \item[\texttt{grit::optimize_dead_defs}] zastřešuje optimalizaci nepoužitých
    definic funkcí a objektů.
  \item[\texttt{grit::optimize_values}] poskytuje optimalizaci známých hodnot.
  \item[\texttt{grit::optimize_inline}] implementuje inlining.
  \item[\texttt{grit::interf}] zkonstruuje pro funkci graf interference.
  \item[\texttt{grit::slot_alloc}] pomocí grafu interference alokuje proměnným
    sloty.
  \item[\texttt{grit::to_asm}] přeloží Grit do interní podoby assembleru.
  \item[\texttt{grit::to_sexpr}] převede Grit na s-výraz.

  \item[\texttt{asm::syntax}] definuje interní syntaxi assembleru.
  \item[\texttt{asm::simplify}] implementuje jednoduchou optimalizaci na úrovni
    instrukcí.
  \item[\texttt{asm::to_gas}] generuje vstup pro GNU Assembler.
\end{description}

\section{Běhová knihovna}

Běhová knihovna je implementována v~\Cplusplus{} a má rozsah téměř 2000 řádků.
\Cplusplus{} bylo místo čistého C použito především kvůli podpoře jmenných
prostorů a \texttt{auto}-deklarace proměnných. Možnosti jazyka jdoucí nad rámec
konzervativního C (výjimky, RTTI, virtuální funkce, ...) jsou záměrně
potlačeny\footnote{Pomocí přepínačů \texttt{-fno-rtti -fno-exceptions}
překladače \texttt{clang++}.} a použití standardní knihovny je omezeno na její
část odpovídající jazyku C.

\section{Testy}

Jednotkové testy kontrolující funkci vybraných částí překladače jsou umístěny
přímo v~kódu pomocí nástrojů podporovaných přímo jazykem Rust. Hlavní porce
testů má podobu malých programů, které jsou automaticky přeloženy a spuštěny.
Jejich výstup je poté porovnán s~očekávaným výstupem a pokud se liší, je
ohlášena chyba. Většina z~objemu testů testuje jednotlivé oblasti standardní
knihovny, další větši skupinou jsou zátěžové testy, které jsou převážně zaměřené
na testování správy paměti.
