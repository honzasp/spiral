\documentclass[a4paper,9pt,oneside,twocolumn,article]{memoir}
\usepackage[english]{babel}
\usepackage[utf8]{inputenc}
\usepackage[T1]{fontenc}
\usepackage{hyperref}
\usepackage[fixlanguage]{babelbib}
\usepackage{geometry}

\selectbiblanguage{english}
\def\Cplusplus{{C\nolinebreak[4]\hspace{-.05em}\raisebox{.4ex}{\tiny\bfseries ++}}}

\geometry{
  marginparsep=0cm,marginparwidth=0cm,
  inner=1.5cm,outer=1.5cm,
  top=0cm,bottom=1.5cm,
}

\hypersetup{
  unicode=true,
  pdftitle={Standard library data structures for Spiral},
  pdfauthor={Jan Špaček},
  hidelinks,
}

\setbeforesecskip{0.1cm}
\setaftersecskip{0.1cm}
\setlength{\beforechapskip}{0.1cm}
\setlength{\afterchapskip}{0.1cm}
\setlength{\bibitemsep}{0.0cm}

\begin{document}

\title{Standard library data structures for Spiral}
\date{April 2016}
\author{Jan Špaček}
\maketitle

\chapter{Introduction}

Spiral \cite{spiral} is an untyped impure functional language. Its standard
library provides various primitive data structures (cons cells, tuples, strings,
arrays), implemented in the runtime library, but it lacked any higher-level data
structures, like finite maps, sets or heaps. We present a toolbox of data
structures implemented in Spiral that extend the library with persistent maps,
sets and heaps and mutable hash tables. We have also added some new primitives
(symbols and mutable references) and fixed a handful of bugs in the runtime.

\chapter{Persistent maps}

Persistent maps are located in module \texttt{std.map}. The data structure is
implemented as an AA~tree \cite{andersson1993balanced}, which is a~binary
representation of a~2,3-tree, where only right edges can be horizontal. AA~trees
need to store the rank in every node and are rebalanced using two operations,
\texttt{skew} (remove horizontal left edges) and \texttt{split} (remove two
consecutive horizontal right edges).

\section{Red-black trees}

There is also an alternative implementation of persistent maps in module
\texttt{std.redmap}, employing red-black trees. Very compact functional
insertion into red-black trees is presented in  \cite{okasaki1999purely}. The key
idea is that a~tree with red root and one red child can be trivially restored to
a~valid red-black tree by blackening the root. By temporarily allowing this
violation of the invariant, we can implement \texttt{insert} recursively by
propagating the red color up in the tree.

Functional deletion was given 15~years later in \cite{germane2014deletion}.
Similar to insertion, deletion is often described as propagating an extra black.
We can introduce a~third color, black-black, and rebalance the tree in
a~recursive bottom-up fashion, as we did for insertion. There are more cases to
consider, but they are relatively straightforward. This is in contrast to
an~effective imperative implementation, as described in
\cite{thomas2001introduction}, where the extra black is only conceptual.

\chapter{Persistent sets}

Persistent sets, residing in module \texttt{std.set}, use weight-balanced binary
trees \cite{adams1992implementing}. Our implementation was inspired by the
\texttt{containers} package in Haskell \cite{containers}. The tree is balanced
to keep the ratio of the weights of the left and right subtree of every node
bounded, so a~bottom-up recursive implementation is very natural. One advantage
of this balancing scheme is that the sizes of the trees can be used to
effectively index into the set.

\chapter{Persistent heaps}

Our implementation of persistent heaps from module \texttt{std.heap} uses
weight-biased variant of leftist heaps described in \cite{okasaki1999purely}.
The weight-biased leftist property asserts that the size of a~right subtree must
not be greater than the size of a~left subtree, so the right spine of the tree
is at most logarithmic in the size of the tree. Heaps can therefore be
efficiently merged along their right spines. The main reason for balancing on
the weight instead of the rank (length of right spine) is that we have constant
time access to the number of elements in the heap.

\chapter{Hash maps}

Hash maps are typical data structures that absolutely require mutation, thus
cannot be (reasonably) implemented in a pure functional way. Fortunately, impure
data structures are as welcome in Spiral as the pure ones. The hash map from
module \texttt{std.hashmap} uses Robin Hood hashing \cite{celis1986robin} with
linear probing \cite{robinhood,robinhooddel}, as does \texttt{HashMap} from the
Rust standard library \cite{ruststd}.

The idea of Robin Hood hashing is that upon insertion, the buckets are linearly
probed until an empty bucket is found. However, if we find that the inserted
entry is further away from its initial bucket than the entry in the probed
bucket, we swap them and continue the process with the displaced entry. After
removing an entry, we shift the following entries backward until we hit an empty
bucket or an entry that resides in its initial bucket (has probe length 0).

\section{Hashing}

Our hashing function is SipHash-2-4 \cite{aumasson2012siphash}, which is claimed
to be cryptographically safe. The core algorithm is implemented in the runtime
written in \Cplusplus{}, because hashing must be fast and requires raw integer
operations, none of which is available in Spiral. The Spiral interface to the
hasher is exported from \texttt{std.hash}.

\chapter{Implementation details}

Aside from hashing, we also extended the runtime to support mutable references
and symbols. References are mutable boxes that contain exactly one value, which
can be used in imperative algorithms as mutable variables (we needed them for
hash maps). Symbols are unique objects that can be used as markers or keys in
containers. For convenience, symbols wrap a value (usually a string), mostly as
an aid for debugging. At the moment, symbols are used to represent color in
red-black trees and as markers in data structures (they are represented as
tuples, where the first element is the symbol with the name of the structure).

\chapter{Conclusion}

We were surprised that even though Spiral is very immature, we found the
language quite expressive. The biggest obstacle is the poor error reporting:
Spiral itself collects no stack traces, so a~C~debugger was essential in
debugging the programs. Unfortunately, the non-standard calling convention and
aggressive inlining performed by the Spiral compiler often made the stack traces
unusable, so debugging often resorted to sprinkling the code with
\texttt{println}s.

{\small \raggedright
\bibliographystyle{babplain}
\bibliography{refs}}

\end{document}
