\chapter{Implementation}

The complete implementation is available in a git repository from GitHub at
\url{https://github.com/honzasp/spiral}. The sources of the compiler are placed
in directory \texttt{src/}, the runtime library is located inside \texttt{rt/}, the
directory \texttt{stdlib/} contains the standard library modules and
\texttt{tests/} collects functional tests. All files are released into public
domain (see file \texttt{UNLICENSE}).

\section{Compiler}

The compiler is written in Rust \cite{rust} and with more than 6000 lines of
code\footnote{Only lines with at least two non-spaces were counted.} (including unit
tests) is the biggest component. The program is divided into modules that
mirror the sequence of intermediate languages used during translation.

\begin{description}
  \item[\texttt{main}] defines the function `main` that performs the actions
    entered by the user on the command line.
  \item[\texttt{args}] is responsible for decoding the command line arguments.
\end{description}
~
\begin{description}
  \item[\texttt{sexpr::syntax}] defines the data types for s-expressions.
  \item[\texttt{sexpr::parse}] provides an s-expression parser.
  \item[\texttt{sexpr::pretty_print}] transforms s-expressions from the internal
    form to a human-readable textual representation.
  \item[\texttt{sexpr::to_spiral}] extracts the syntax tree of Spiral from
    an s-expression.
  \item[\texttt{sexpr::to_spine}] converts an s-expression to Spine program
    (used for testing purposes).
  \item[\texttt{sexpr::to_grit}] reads an s-expression as a Grit program (also
    used for testing).
\end{description}
~
\begin{description}
  \item[\texttt{spiral::syntax}] defines the Spiral syntax tree.
  \item[\texttt{spiral::env}] defines a helper environment for traversing the
    syntax tree.
  \item[\texttt{spiral::imported}] collects all imported modules from a module
    or a program.
  \item[\texttt{spiral::to_spine}] translates a program from Spiral to Spine,
    using:
  \item[\texttt{spiral::to_spine::mods}] to translate modules,
  \item[\texttt{spiral::to_spine::decls}] to translate declarations,
  \item[\texttt{spiral::to_spine::stmts}] to translate statements, and
  \item[\texttt{spiral::to_spine::exprs}] to translate expressions.
\end{description}
~
\begin{description}
  \item[\texttt{spine::syntax}] defines the Spine syntax tree.
  \item[\texttt{spine::env}] provides the environment for traversing the tree.
  \item[\texttt{spine::check}] implements a well-formedness checker (used for
    testing).
  \item[\texttt{spine::eval}] is a simple evaluator (interpreter) of Spine
    programs. It played a major role during development, because it allowed
    us to write and test the translation to Spine before starting the work on
    the following languages.
  \item[\texttt{spine::free}] discovers all free variables of a function. These
    variables must be captured in the definition of the function.
  \item[\texttt{spine::onion}] defines the ,,onion'' for translating from
    Spiral.
  \item[\texttt{spine::to_grit}] translates a program from Spine to Grit.
  \item[\texttt{spine::to_sexpr}] converts Spine into an s-expression.
\end{description}
~
\begin{description}
  \item[\texttt{grit::syntax}] defines the syntax of Grit.
  \item[\texttt{grit::optimize_dead_vals}] implements dead value optimization.
  \item[\texttt{grit::optimize_dead_defs}] contains the dead code optimization
    that removes definitions of unused functions and objects.
  \item[\texttt{grit::optimize_values}] defines the known value optimization.
  \item[\texttt{grit::optimize_inline}] implements inlining.
  \item[\texttt{grit::interf}] builds an interference graph of a function.
  \item[\texttt{grit::slot_alloc}] allocates slots for variables using the
    interference graph.
  \item[\texttt{grit::to_asm}] translates Grit to assembler in the internal
    form.
  \item[\texttt{grit::to_sexpr}] converts Grit to an s-expression.
\end{description}
~
\begin{description}
  \item[\texttt{asm::syntax}] defines the internal form of assembler.
  \item[\texttt{asm::simplify}] implements simple instruction-level
    optimizations.
  \item[\texttt{asm::to_gas}] generates input for GNU Assembler.
\end{description}

\section{Runtime library}

The runtime library is implemented in \Cplusplus{} in less than 2000 lines. The
main reasons that lead us to use \Cplusplus{} instead of pure C were namespaces and
\texttt{auto}-declarations. The features of the language that require hidden
code generated by the compiler (exceptions, RTTI, virtual methods...) are
avoided and, where possible, disabled\footnote{Using the \texttt{clang++} flags
\texttt{-fno-rtti} and \texttt{-fno-exceptions}}. We use only the C standard
library.

\section{Tests}

Unit tests covering the compiler are placed directly in the code using the tools
provided by Rust. We test the whole implementation by compiling and running a
set of small Spiral programs. We then compare the real output with expected
output and report an error if they differ. Most of the tests focus on the
standard library, there are also tests checking that the garbage collector works
correctly.
