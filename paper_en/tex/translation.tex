\chapter{The translation pipeline}

The compiler parses the source code into a general s-expression structure and
transforms it into the Spiral syntax tree. All modules and the main program are
merged and translated into continuation-passing style in the first intermediate
language Spine. The next step converts Spine into the next intermediate
language, imperative Grit. The optimization passes then operate on Grit and then
the code generator emits assembly.

\section{S-expressions}

S-expressions are read using a simple hand-written parser. The resulting data
structure serves as an input to further processing. There is also a
pretty-printer to translate the s-expression back to textual form. Spine and
Grit can be read from s-expressions and written to them for the purposes of
testing and debugging.

\section{Spiral}

Syntax tree of Spiral is decoded from s-expression using a simple but tedious
process. If the program contains syntax errors, the compiler detects them in
this phase and rejects the program. Imported modules are collected in a pass
through the tree, loaded and also examined in the same way. Then the compiler
computes a topological ordering of the modules for further processing, producing
an error if the dependency graph is cyclic.

\section{Spine}

The first intermediate language is Spine. This language is derived from
$\lambda^U_\text{CPS}$ \cite{kennedy2007compiling} and based on
$\lambda$-calculus and continuation passing style.

Continuation is a special $\lambda$-value that cannot escape the local function
and never returns. Calling a continuation is equivalent to jumping to a basic
block in an imperative language or SSA form (single static assignment).
Functions take a return continuation as an argument and return value to the
caller by jumping to this continuation.

\begin{gram}
<program>   = (program <cont-name> <term>)
<term>      = (letcont <cont-def>... <term>)
            | (letfun <fun-def>... <term>)
            | (letobj <obj-def> <term>)
            | (call <val> <cont-name> <val>...)
            | (extern-call <extern-name> <cont-name> <val>...)
            | (cont <cont-name> <val>...)
            | (branch <boolval> <cont-name> <cont-name>)
<fun-def>   = (<var> <cont-name> (<var>...) (<var>...) <term>)
<cont-def>  = (<cont-name> (<var>...) <term>)
<obj-def>   = (string <var> <string-literal>)
            | (double <var> <double-literal>)
<val>       = <var> | <int-literal> | (true) | (false)
<boolval>   = (is-true <val>) | (is-false <val>)
\end{gram}

A program is defined by a term and a halting continuation. Jump to the halting
continuation terminates the program.

\begin{description}
\item[\texttt{(cont <cont-name> <arg>...)}] jumps to the given continuation,
  passing the given arguments. The number of arguments on the call site must
  match the definition of the continuation..

\item[\texttt{(branch <boolval> <then-name> <else-name>)}] evaluates a boolean
  value and jumps to one of the passed continuations. Both continuations must
  expect zero arguments.

\item[\texttt{(call <fun-val> <return-cont> <arg>...)}] calls the function with
  the given arguments, passing \texttt{<return-cont>} as the return
  continuation. This is a tail call If the continuation is also the caller's
  return continuation.

\item[\texttt{(extern-call <extern-name> <return-cont> <arg>...)}] calls an
  extern function by its name and passes its result to \texttt{<return-cont>}.
  Extern calls are never translated as tail calls.

\item[\texttt{(letcont (<cont-name> (<arg>...) <body>)... <term>)}] defines
  a group of mutually recursive continuations.

\item[\texttt{(letfun (<fun> <ret-cont> (<capture>...) (<arg>...) <body>)...
  <term>)}] defines mutually recursive functions. The function can use variables
  visible at the definition, but must list them in the capture list.
  However, no continuations from the outer context are available in the
  function.

\item[\texttt{(letobj <obj-def> <term>)}] defines an object (string or a real
  number).
\end{description}

All values (\texttt{<val>}) are atomic (variables or constants) and can be
duplicated with no restriction, as their evaluation is free.

\subsection{Překlad z~jazyka Spiral}

Při překladu výrazu z~jazyka Spiral do Spine je třeba převést program
z~\emph{direct style} do \emph{continuation-passing style}. Základem překladu
jsou dvě funkce, \texttt{translate-expr(spiral-expr) -> (onion, spine-val)} a
\texttt{translate-expr-tail(spiral-expr, spine-cont-name) -> spine-term}.

Funkce \texttt{translate-expr} přeloží předaný výraz ve Spiral tak, že vrátí
"cibuli"\footnote{Na obranu komunity počítačových vědců je třeba dodat, že
tento název vymyslel autor sám.} (\emph{onion}) a hodnotu. "Slupka cibule" je
tvořena výrazy \texttt{letcont}, \texttt{letfun} a \texttt{letobj}. Uvnitř této
slupky je výsledek výrazu reprezentován danou hodnotou.

Druhá funkce \texttt{translate-expr-tail} pak přeloží výraz do podoby výrazu ve
Spine, který s~výslednou hodnotou skočí do předané \emph{continuation}. Tímto
způsobem jsou přeložena například koncová volání (\emph{tail-calls}).

Pro ilustraci si ukážeme příklad, jak je přeložena tato funkce ve Spiral:

\begin{spiral}
(fun big-enough? (x)
  (if (< x 0)
    (println "small")
    (println "ok")))
\end{spiral}

Nejprve vygenerujeme jméno návratové \emph{continuation} pro funkci
\texttt{big-enough?}, například \texttt{r}. Tělo funkce musí být přeloženo tak,
aby volání v~koncových pozicích byla koncová, tedy aby vracela do \texttt{r}.
Proto na výraz \texttt{(if (< x 0) (println "small") (println "ok"))} použijeme
\texttt{translate-expr-tail} s~\texttt{r}.

Při vyhodnocení výrazu \texttt{if} musíme nejprve vyhodnotit podmínku \texttt{(<
x 0)}. Tu přeložíme pomocí \texttt{translate-expr}, čímž dostaneme "cibuli"
\texttt{(letcont (lt-ret (lt-result) ?) (call < lt-ret x 0))}. Otazník označuje
místo, kde obdržíme výsledek v~proměnné \texttt{lt-result} ("díru").

Obě alternativy v~příkazu \texttt{if} přeložíme opět pomocí
\texttt{translate-expr-tail} s~\emph{continuation} \texttt{r}, abychom zachovali
koncová volání. Obdržíme Spine výrazy \texttt{(letobj (string s1 "small") (call
println r s1))} a \texttt{(letobj (string s2 "ok") (call println r s2))}.

Samotné větvení provedeme výrazem \texttt{branch}, na to ale potřebujeme
\emph{continuation}, do které můžeme skočit. Ty si proto vytvoříme (nazveme je 
\texttt{on-true} a \texttt{on-false}) a vložíme do nich přeložené výrazy z~obou
větví zdrojového výrazu \texttt{if}. Výsledek pak vypadá takto:

\begin{spine}
(letcont (lt-ret (lt-result)
            (letcont (on-true ()
                        (letobj (string s1 "small") (call println r s1)))
                     (on-false ()
                        (letobj (string s2 "ok") (call println r s2)))
              (branch (is-true lt-result) on-true on-false)))
  (call < lt-ret x 0))
\end{spine}

Je vidět, že pořadí vyhodnocování a přenos informací z~původního programu je
nyní explicitně vyjádřeno. Nejprve se zavolá funkce \texttt{<} s~argumenty
\texttt{x} a \texttt{0}. Ta svůj výsledek předá v~proměnné \texttt{lt-result} do
\texttt{lt-ret}. Ta následně tuto proměnnou prozkoumá, a pokud je její hodnota
pravdivá, skočí do \texttt{on-true}, jinak do \texttt{on-false}.
V~\texttt{on-true} a \texttt{on-false} pak definujeme patřičný řetězec a
následně zavoláme funkci \texttt{println}, které tento řetězec předáme. Funkce
\texttt{println} vrátí svůj výsledek do \texttt{r}, což je ale návratová
\emph{continuation} volající funkce \texttt{big-enough?}, takže toto volání bude
přeloženo jako koncové.

\section{Grit}

Dalším jazykem v~pořadí je Grit. Tento jazyk je již poměrně nízkoúrovňový a
blízký assembleru. Definice všech funkcí a objektů jsou globální a proměnné jsou
zapisovatelné a jsou pojmenované pouze čísly. Funkce se skládají z~bloků, které
obsahují sekvenci operací a jsou ukončeny skokem.

\begin{gram}
<program>   = (program <fun-name> <fun-def>... <obj-def>...)
<fun-def>   = (fun <fun-name> <int> <int> <int> <label> <block>...)
<obj-def>   = (string <obj-name> <string-literal>)
            | (double <obj-name> <double-literal>)
<block>     = (<label> <op>... <jump>)

<op>        = (call <var> <callee> <val>...)
            | (extern-call <var> <extern-name> <val>...)
            | (alloc-clos (<var> <fun-name> <val>...)...)
            | (assign (<var> <val>)...)
<jump>      = (goto <label>)
            | (return <val>)
            | (tail-call <callee> <val>...)
            | (branch <boolval> <label> <label>)
<callee>    = (combinator <fun-name>)
            | (known-closure <fun-name> <val>)
            | (unknown <val>)

<val>       = (var <int>)
            | (arg <int>)
            | (capture <int>)
            | (combinator <fun-name>)
            | (obj <obj-name>)
            | (int <int>)
            | (true)
            | (false)
            | (undefined)
<boolval>   = (is-true <val>) | (is-false <val>)
\end{gram}

Jednotlivé operace reprezentují vše, co program může vykonávat:

\begin{description}
  \item[\texttt{(call <var> <callee> <arg>...)}] zavolá funkci specifikovanou
    v~\texttt{<callee>} se zadanými argumenty a její výsledek zapíše do proměnné.
    Hodnoty \texttt{<callee>} mohou být:
    \begin{description}
      \item[\texttt{(combinator <fun-name>)}] je volání kombinátoru, tedy
        funkce, která nemá žádné uložené proměnné z~prostředí. Taková funkce je
        staticky alokovaná, není proto třeba žádná proměnná na její uchování.
        U tohoto volání je navíc možno zkontrolovat počet argumentů předem,
        takže volaná funkce je již kontrolovat nemusí.
      \item[\texttt{(known-closure <fun-name> <val>)}] je volání známé funkce,
        která má uložené proměnné z~prostředí (\emph{closure}), takže ji musíme
        volat spolu s~její funkční hodnotou. Pokud by hodnota \texttt{<val>}
        nebyla funkce \texttt{<fun-name>}, program by selhal. I zde je zaručeno,
        že počet argumentů odpovídá, takže volaná funkce je nekontroluje.
      \item[\texttt{(unknown <val>)}] je volání, u kterého není staticky
        zjištěno, kterou funkci bude volat. Za běhu je proto nutno zkontrolovat,
        že \texttt{<val>} je skutečně funkce, a tato pak ještě sama kontroluje
        počet předaných argumentů.
    \end{description}

  \item[\texttt{(extern-call <var> <extern-name> <arg>...)}] zavolá externí
    funkci s~předanými argumenty a výsledek zapíše do dané proměnné.

  \item[\texttt{(alloc-clos (<var> <fun-name> <capture>...))}] alokuje funkce se
    zadanými názvy a určeným seznamem zachycených hodnot. Tyto funkce jsou
    zapsány do proměnných ještě předtím, než jsou zachycené hodnoty zapsány, což
    umožňuje funkci uložit referenci na sebe samu. Rovněž je takto možné získat
    hodnotu kombinátoru, pokud je seznam zachycených hodnot prázdný -- v~tom
    případě se nic nealokuje.

  \item[\texttt{(assign (<var> <val>)...)}] zapíše do proměnných odpovídající
    hodnoty. Toto přiřazení proběhne najednou, takže na pravé straně je možno
    použít proměnné z~levé strany, jejichž hodnota bude odpovídat hodnotě před
    operací \texttt{assign}.
\end{description}

Skoky mohou být rovněž několika druhů:

\begin{description}
  \item[\texttt{(goto <label>)}] skočí na daný blok.
  \item[\texttt{(return <val>)}] vrátí z~funkce určenou hodnotu.
  \item[\texttt{(tail-call <callee> <arg>...)}] provede koncové volání
    \texttt{<callee>} (podobně jako v~operaci \texttt{call}) s~argumenty. Tímto
    se místo na zásobníku zabrané volající funkcí uvolní ještě před vstupem do
    volané funkce.
  \item[\texttt{(branch <boolval> <then> <else>)}] skočí na jeden z~bloků podle
    určené pravdivostní hodnoty.
\end{description}

Paleta hodnot je širší než ve Spine, ovšem stále platí, že každá hodnota je
konstantní nebo ji lze snadno načíst z~paměti:

\begin{description}
  \item[\texttt{(var <index>)}] je hodnota proměnné.
  \item[\texttt{(arg <index>)}] je hodnota argumentu.
  \item[\texttt{(capture <index>)}] je hodnota zachycené hodnoty z~aktuální funkce.
  \item[\texttt{(combinator <fun-name>)}] je konstantní hodnota staticky
    alokovaného kombinátoru.
  \item[\texttt{(obj <obj-name>)}] je konstantní hodnota staticky alokovaného
    objektu.
  \item[\texttt{(int <int>)}] je celočíselná konstanta.
  \item[\texttt{(true)}, \texttt{(false)}] jsou pravdivostní konstanty.
  \item[\texttt{(undefined)}] je nedefinovaná hodnota. Tuto hodnotu mohou
    vytvořit optimalizace, například při čtení z~proměnné, do které nebylo
    zapsáno. Při zápisu nedefinované hodnoty do paměti či registru během
    překladu do~assembleru není vygenerována žádná instrukce.
\end{description}

\subsection{Překlad z~jazyka Spine}

Překlad ze Spine do Gritu je poměrně přímočarý. Volání \emph{continuation} je
přeloženo jako přiřazení do proměnných vygenerovaných jako její argumenty a skok
do bloku, kterým tělo \emph{continuation} začíná. Ostatní konstrukce ve Spine
mají v~jazyku Grit přímý ekvivalent.

\subsection{Optimalizace}

Jelikož je jazyk Spiral poměrně chudý, nabízí se relativně málo možností
optimalizace. Proto se první optimalizační fáze aplikují až na nízkoúrovňový
Grit a jedná se převážně o zjednodušující a zeštíhlující proces.

\subsubsection{Optimalizace známých hodnot}

V~této fázi, která operuje nad celým programem, je určen odhad hodnot, kterých
může nabýt každá proměnná, zachycená proměnná a návratová hodnota. Tyto
informace jsou poté využity několikerým způsobem:

\begin{itemize}
  \item Optimalizace známých volání nahrazením \texttt{(call (unknown ...) ...)}
    za \texttt{(call (known-closure ...) ...)} nebo \texttt{(call (combinator
    ...) ...)}.
  \item Nahrazení větvení nepodmíněným skokem, pokud je možné dokázat, že
    testovací hodnota je vždy pravdivá nebo vždy nepravdivá.
  \item Propagace konstant odstraní proměnné, jejichž hodnota je konstantní
    (číslo, pravdivostní hodnota nebo staticky alokovaný objekt).
\end{itemize}

\subsubsection{Odstranění nepotřebných hodnot}

Tato fáze je opět globální a jejím účelem je převážně zrušit nepotřebné
zachycené hodnoty, ovšem zároveň je možno odstranit i zápisy do nevyužitých
proměnných a alokaci zbytečných funkcí. Často se stane, že funkce přijde o
všechny zachycené hodnoty, čímž se stane kombinátorem. Její použití pak již není
spojeno s~alokací paměti.

\subsubsection{Inlining funkcí}

Během fáze inliningu jsou volání vybraných funkcí nahrazena přímým vložením
volané funkce do funkce volající. Tato optimalizace je pro funkcionální programy
zcela klíčová, protože se většinou skládají z~velkého počtu malých funkcí.
Jejich expanzí na místo volání se ušetří instrukce pro manipulaci se zásobníkem
a pro samotné volání a návrat z~funkce. Zároveň je pak možné provést další
optimalizace, převážně proto, že jsou nyní zpravidla k~dispozici lepší informace
o argumentech volané funkce. Na druhou stranu takto dochází k~duplikaci kódu a
většinou tak i k~nárůstu délky výsledného programu.

Pro inlining jsou nyní vybírány funkce, které jsou dostatečně malé, jsou
kombinátory a nevolají jiné funkce kromě externích. Do této kategorie spadá
většina funkcí standardní knihovny, které obvykle pouze obalují externí volání
podpůrné běhové knihovny.

\subsubsection{Odstranění nepotřebných definic}

Tato optimalizace je známá rovněž jako odstranění mrtvého kódu (\emph{dead code
elimination}) a spočívá ve vynechání funkcí a staticky alokovaných objektů,
které nejsou referencovány z~hlavní funkce a jsou tedy zbytečné. Kromě
nepoužitých definic z~importovaných modulů jsou obvykle odstraněny i funkce,
jejichž volání byla všude inlinována.

\subsubsection{Pořadí optimalizací}

U optimalizačních fází je velmi důležité pořadí, v~jakém jsou na program
aplikovány. Optimalizace známých hodnot zanechá část proměnných a zachycených
hodnot bez využití. Tyto proměnné jsou následně odstraněny ve fázi odstranění
nepotřebných hodnot. Zároveň se takto zvýší počet kombinátorů, které můžeme
inlinovat. Po inliningu pak dostaneme řadu nepotřebných definic, které jsou
kandidáty na odstranění.

Úroveň optimalizací je možno ovlivnit z~argumentů příkazové řádky. Na úrovni 0
neprobíhá žádná optimalizace. Úroveň 1 zapne všechny optimalizace ve výše
uvedeném pořadí kromě inliningu, který je umožněn až od úrovně 2. Při úrovni 3
pak po inliningu následuje ještě jednou optimalizace známých hodnot a odstranění
nepotřebných hodnot.

\section{Alokace slotů}

Program v~Gritu typicky pracuje s~velkým množstvím proměnných. Bylo by ovšem
značným plýtváním pamětí, pokud bychom každé proměnné alokovali na zásobníku
zvláští prostor (slot), jelikož doba života proměnné je obvykle krátká a její
slot by byl většinu času nevyužit. Proto je žádoucí několika proměnným přiřadit
stejný slot, ovšem samozřejmě tak, aby nedošlo ke ztrátě informace, tedy aby
zápis do proměnné nepřepsal hodnotu jiné proměnné, ze které se bude později
číst.

Této fázi v~běžných překladačích odpovídá alokace registrů, ovšem náš generátor
kódu je zjednodušený, takže všechny proměnné umisťuje do paměti. Podobně jako
při alokaci registrů však můžeme použít přístup s~užitím barvení grafu
interference
\cite{chaitin1981register,chaitin1982register,briggs1994improvements}, ovšem
postup je jednodušší díky tomu, že počet barev (registrů) nemáme omezen, pouze
se jej snažíme minimalizovat. Proto stačí pouze sestavit graf interference a ten
pak hladovým algoritmem obarvit. U hladového algoritmu je klíčové pořadí,
v~jakém jsou vrcholy barveny. V~našem případě vrcholy barvíme v~pořadí podle
počtu vycházejících hran sestupně.

\section{Assembler}

Posledním krokem překladu je generování assembleru pro architekturu IA-32
z~jazyka Grit. Výsledný soubor obsahuje kód všech funkcí, staticky alokované
objekty a řetězce.  Kvůli podpoře koncových volání mají funkce jazyka Spiral
jinou volací konvenci (\emph{calling convention}) než funkce jazyka C. Argumenty
pro funkci v~C jsou totiž umístěny na zásobníku v~oblasti volající funkce, pro
koncová volání je však potřeba, aby z~volající funkce na zásobníku nic nezbylo,
argumenty proto musí volaná funkce dostat do své oblasti.

Zásobník ve funkci s~\texttt{N} sloty vypadá takto (relativně vůči vrcholu
zásobníku v~registru \texttt{\%esp}):

\begin{ttcode}
  4*N+4  :  return address
    4*N  :  slot 0 (argument 0)
  4*N-4  :  slot 1 (argument 1)
            ...
      8  :  slot (N-2)
      4  :  slot (N-1)
      0  :  closure value
\end{ttcode}

Volaná funkce dostane argumenty ve svých prvních slotech, které se nachází těsně
pod návratovou adresou (zapsanou instrukcí \texttt{call}). Ostatní sloty jsou
umístěny níž. Hodnota funkce je umístěna v~registru \texttt{\%ecx}, počet
argumentů u neznámých volání v~\texttt{\%eax}. Hodnotu z~\texttt{\%ecx} funkce
zapíše na začátek svého zásobníku, aby bylo možno projít zásobník při úklidu
paměti.  Návratovou hodnotu pak funkce zpět na místo volání předá v~registru
\texttt{\%eax}.

Volání funkce proběhne tak, že se argumenty umístí pod prostor obsazený volající
funkcí a instrukcí \texttt{call} se na zásobník uloží návratová adresa.  Při
koncovém volání argumenty přepíšou sloty volající funkce, zásobník se posune
zpátky nahoru a do funkce se provede skok (instrukcí \texttt{jmp}). Návratová
adresa tak zůstane nezměněna a při návratu z~koncové funkce dojde ke skoku do
původní volající funkce.

Jelikož všechny proměnné jsou uloženy v~paměti, může generátor kódu použít pevně
dané registry pro dočasné uložení hodnot. Registry \texttt{\%eax} a
\texttt{\%edx} se používají při přesunech hodnot v~paměti, v~registru
\texttt{\%ecx} je uložena hodnota aktuální funkce, která se využívá při
přístupu k~zachyceným hodnotám.  V~jistých speciálních případech je rovněž
využit i registr \texttt{\%ebx}. Dále funkce přistupuje k~registru
\texttt{\%edi}, ve kterém je uložen ukazatel na aktuální pozadí
(\texttt{Bg*}) běhové knihovny.
