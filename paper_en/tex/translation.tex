\chapter{The translation pipeline}

The compiler parses the source code into a general s-expression structure and
transforms it to the Spiral syntax tree. All modules and the main program are
then merged and translated to the first continuation-passing style intermediate
language Spine. The next step converts Spine into the imperative intermediate
language Grit. The optimization passes operate on Grit and then the code
generator emits assembly.

\section{S-expressions}

S-expressions are read using a simple hand-written parser. The resulting data
structure serves as an input for further processing. There is also a
pretty-printer to translate the s-expression back to textual form. Spine and
Grit can be read from s-expressions and written to them for testing and
debugging purposes.

\section{Spiral}

The syntax tree of Spiral is decoded from an s-expression using a simple but tedious
process. If the program contains syntax errors, the compiler detects them in
this phase and rejects the program. Imported modules are collected in a pass
through the tree, loaded and also examined in the same way. Then the compiler
computes a topological ordering of the modules for further processing, reporting
an error if the dependency graph is cyclic.

\section{Spine}

The first intermediate language is Spine. This language is derived from
$\lambda^U_\text{CPS}$ \cite{kennedy2007compiling} and based on
$\lambda$-calculus and continuation passing style.

A continuation is a special $\lambda$-value that cannot escape the local
function and never returns. Calling a continuation is equivalent to jumping to a
basic block in an imperative language or in SSA form (single static assignment).
Functions take a return continuation as a special argument and return value to
the caller by jumping to this continuation.

\begin{gram}
<program>   = (program <cont-name> <term>)
<term>      = (letcont <cont-def>... <term>)
            | (letfun <fun-def>... <term>)
            | (letobj <obj-def> <term>)
            | (call <val> <cont-name> <val>...)
            | (extern-call <extern-name> <cont-name> <val>...)
            | (cont <cont-name> <val>...)
            | (branch <boolval> <cont-name> <cont-name>)
<fun-def>   = (<var> <cont-name> (<var>...) (<var>...) <term>)
<cont-def>  = (<cont-name> (<var>...) <term>)
<obj-def>   = (string <var> <string-literal>)
            | (double <var> <double-literal>)
<val>       = <var> | <int-literal> | (true) | (false)
<boolval>   = (is-true <val>) | (is-false <val>)
\end{gram}

A program is defined by a term and a halting continuation. A jump to the halting
continuation terminates the program.

\begin{description}
\item[\texttt{(cont <cont-name> <arg>...)}] jumps to the given continuation,
  passing the given arguments. The number of arguments on the call site must
  match the definition of the continuation.

\item[\texttt{(branch <boolval> <then-name> <else-name>)}] evaluates a boolean
  value and jumps to one of the passed continuations. Both continuations must
  expect zero arguments.

\item[\texttt{(call <fun-val> <return-cont> <arg>...)}] calls the function with
  the given arguments, passing \texttt{<return-cont>} as the return
  continuation. This is a tail call if the continuation is also the caller's
  return continuation.

\item[\texttt{(extern-call <extern-name> <return-cont> <arg>...)}] calls an
  extern function by its name and passes its result to \texttt{<return-cont>}.
  Extern calls are never translated to tail calls.

\item[\texttt{(letcont (<cont-name> (<arg>...) <body>)... <term>)}] defines
  a group of mutually recursive continuations.

\item[\texttt{(letfun (<fun> <ret-cont> (<capture>...) (<arg>...) <body>)...
  <term>)}] defines mutually recursive functions. The functions can use variables
  visible at the definition, but must list them in the capture list.
  However, no continuations from the outer context are available in the
  function.

\item[\texttt{(letobj <obj-def> <term>)}] defines an object (a string or a real
  number).
\end{description}

All values (\texttt{<val>}) are atomic (variables or constants) and can be
duplicated without restriction, as their evaluation is free.

\subsection{Translation from Spiral}

Translating expressions from Spiral to Spine requires converting the program
from direct style to continuation-passing style. The translation is driven by a
pair of functions, \texttt{translate-expr :: SpiralExpr -> (Onion, SpineVal)}
and \texttt{translate-expr-tail :: SpineContName -> SpiralExpr -> SpineTerm}.

\texttt{translate-expr} translates the Spiral expression to an
,,onion''\footnote{Blame the author for this name.} and a value.
The onion consists of layers of \texttt{letcont}, \texttt{letfun} and
\texttt{letobj} Spine terms. Inside, the value of the Spiral expression is
evaluated to the returned Spine value.

The function \texttt{translate-expr-tail} translates the Spiral expression
into a Spine term that jumps to the passed continuation with the evaluated value
of the expression. As an example, tail-calls are translated this way.

To illustrate these functions, let us consider the translation of a simple snippet
in Spiral:

\begin{spiral}
(fun big-enough? (x)
  (if (< x 0)
    (println "small")
    (println "ok")))
\end{spiral}

First, we generate a name for the return continuation of the function
\texttt{big-enough?}, say \texttt{r}. To ensure that the calls in tail positions
will be translated correctly, returning directly to \texttt{r}, we pass the body
of the function (\texttt{(if (< x 0) (println "small") (println "ok"))}) to function
\texttt{translate-expr-tail "r"}.

To evaluate the \texttt{if} expression, we must first evaluate the condition
\texttt{(< x 0)} using \texttt{translate-expr}. We obtain the onion
\texttt{(letcont (lt-ret (lt-result) ?) (call < lt-ret x 0))}, where the
question mark \texttt{?} represents the ,,hole'', the place where we
get the result in variable \texttt{lt-result}.

We translate both arms in the \texttt{if} expression by
\texttt{translate-expr-tail} to preserve the tail calls and get Spine terms
\texttt{(letobj (string s1 "small") (call println r s1))} and \texttt{(letobj
(string s2 "ok") (call println r s2))}.

To generate the \texttt{branch} term, we need two continuations that would serve
as the targets of the conditional jump. We will call these continuations
\texttt{on-true} and \texttt{on-false} and set their bodies to the terms
we have generated from the \texttt{if} arms. The body of the translated function
then looks like this:

\begin{spine}
(letcont (lt-ret (lt-result)
            (letcont (on-true ()
                        (letobj (string s1 "small") (call println r s1)))
                     (on-false ()
                        (letobj (string s2 "ok") (call println r s2)))
              (branch (is-true lt-result) on-true on-false)))
  (call < lt-ret x 0))
\end{spine}

The order of evaluation and the flow of information from the original program
are clearly expressed after the translation to Spine.  We must first call
\texttt{<} with \texttt{x} and \texttt{0}. Then we get the result in
\texttt{lt-result} inside \texttt{lt-ret}, where we scrutinize the variable and
decide which branch will go next. In both branches we must first define the
string that we want to print and then pass it to \texttt{println}. Upon
returning, \texttt{println} will pass its result directly to the callee's return
continuation, so this is a tail-call.

\section{Grit}

Next stage of the pipeline is the language Grit. It is quite low-level and is
close to the assembler. All functions and objects are leveraged to the top level,
variables are mutable and named by integers. Functions are composed from basic
blocks, that contain a list of operations terminated by a jump.

\begin{gram}
<program>   = (program <fun-name> <fun-def>... <obj-def>...)
<fun-def>   = (fun <fun-name> <int> <int> <int> <label> <block>...)
<obj-def>   = (string <obj-name> <string-literal>)
            | (double <obj-name> <double-literal>)
<block>     = (<label> <op>... <jump>)

<op>        = (call <var> <callee> <val>...)
            | (extern-call <var> <extern-name> <val>...)
            | (alloc-clos (<var> <fun-name> <val>...)...)
            | (assign (<var> <val>)...)
<jump>      = (goto <label>)
            | (return <val>)
            | (tail-call <callee> <val>...)
            | (branch <boolval> <label> <label>)
<callee>    = (combinator <fun-name>)
            | (known-closure <fun-name> <val>)
            | (unknown <val>)

<val>       = (var <int>)
            | (arg <int>)
            | (capture <int>)
            | (combinator <fun-name>)
            | (obj <obj-name>)
            | (int <int>)
            | (true)
            | (false)
            | (undefined)
<boolval>   = (is-true <val>) | (is-false <val>)
\end{gram}

The operations represent all actions that the program can do:

\begin{description}
  \item[\texttt{(call <var> <callee> <arg>...)}] calls the function determined
    by \texttt{<callee>} with some arguments and writes the return value to a
    variable. \texttt{<callee>} can be:
    \begin{description}
      \item[\texttt{(combinator <fun-name>)}] calls a combinator, which is a
        function that has no captured variables. This call is very effective,
        because there is no need to store the function value and pass it to the
        callee at runtime. Furthermore, we can check the number of arguments
        during compilation, so the callee does not have to check them at
        runtime.
      \item[\texttt{(known-closure <fun-name> <val>)}] calls a closure that is
        known statically. To generate such a call, the compiler must be able to
        prove that \texttt{<val>} will only ever be a function object of the
        function \texttt{<fun-name>}, otherwise the behavior is undefined (and
        probably catastrophic). We need to store the function value, but we can
        jump directly to the function body and skip the argument check.
      \item[\texttt{(unknown <val>)}] is a fully dynamic call. At runtime, we
        must first check in the caller whether the value is a function and
        report an error otherwise. The callee will then check the number of
        arguments and also report an error if it does not match.
    \end{description}

  \item[\texttt{(extern-call <var> <extern-name> <arg>...)}] calls an external
    function and writes the result into the variable \texttt{<var>}.

  \item[\texttt{(alloc-clos (<var> <fun-name> <capture>...))}] allocates
    closures for functions with given names and list of captured values. The
    variables that hold the closure values are initialized before the captures
    are evaluated, so the functions can reference each other. As a special case,
    if the capture list is empty, no allocation is performed but the static
    function value of the combinator is produced.

  \item[\texttt{(assign (<var> <val>)...)}] rewrites the variables by
    corresponding values. The whole operation is atomic, so a variable assigned
    to on the left-hand side will be evaluated to its former value on the
    right-hand side.
\end{description}

The jumps are formed as following:

\begin{description}
  \item[\texttt{(goto <label>)}] jumps to the given block.
  \item[\texttt{(return <val>)}] returns a value from the function.
  \item[\texttt{(tail-call <callee> <arg>...)}] performs a tail-call to
    \texttt{<callee>} (with the same semantics as in \texttt{call}). The stack
    frame of the current function will be popped before the call.
  \item[\texttt{(branch <boolval> <then> <else>)}] jumps to one of the two
    blocks depending on the boolean value.
\end{description}

There are more types of values than in Spine, but all values are either constant
or reachable by a single load from memory.

\begin{description}
  \item[\texttt{(var <index>)}] is value of a variable.
  \item[\texttt{(arg <index>)}] is value of an argument.
  \item[\texttt{(capture <index>)}] is value of a captured variable.
  \item[\texttt{(combinator <fun-name>)}] is the constant value of a combinator.
  \item[\texttt{(obj <obj-name>)}] is the constant value of a statically
    allocated object.
  \item[\texttt{(int <int>)}] is an integer constant.
  \item[\texttt{(true)}, \texttt{(false)}] are boolean constants.
  \item[\texttt{(undefined)}] is an undefined value. This special value can be
    produced by optimizations, for example as a result of a read from
    uninitialized variable. During the code generation, if an undefined value is
    assigned to a register or a memory location, no code is generated.
\end{description}

\subsection{Translation from Spine}

Translation from Spine to Grit is straightforward. Call to a continuation is
translated as an assignment to variables generated as its arguments followed by
a jump to the first basic block of the continuation. Other structures from Spine
have a direct counterpart in Grit.

\subsection{Optimization}

Because Spiral has very few features, there are not many opportunities for
optimization. The first optimization phases thus operate on the low-level Grit
and usually just simplify the code.

\subsubsection{Known-value optimization}

This whole-program phase first estimates the set of possible values of each
variable, each captured variable and result of every function. This information
is then used at several places:

\begin{itemize}
  \item Known call optimization by substituting \texttt{(call (unknown ...)
    ...)} for \texttt{(call (known-closure ...) ...)} or \texttt{(call
    (combinator ...) ...)}.
  \item Reduction of branches with a condition that is always true or
    always false.
  \item Constant propagation removes variables that have a simple constant value
    (number, boolean or a statically allocated object).
\end{itemize}

\subsubsection{Dead value elimination}

This phase is also global and its main goal is to remove unused captured
variables, but unnecessary allocations and variables are also discarded. Many
functions lose all their captured variables and become combinators, so
the memory consumption and thus the pressure on the garbage collector at runtime is
reduced.

\subsubsection{Function inlining}

Inlining substitutes calls to selected functions by including the body of the
callee into the caller. This is a key optimization for functional programs,
because they are usually composed from many small functions. When functions are
expanded to the call site, we avoid the overhead of manipulating the stack and
two jumps. Further optimization possibilities are also exposed, because the
compiler can gather more information from the local context. The downside is the
possible increase in code side.

We currently inline functions that are small combinators and call no other
functions except external calls. This rule applies to most standard library
functions, as they are usually just thin wrappers around the runtime library.

\subsubsection{Dead code elimination}

This optimization removes the functions and static objects that are not
transitively referenced from the main function. These functions and objects can
never be used, so there is no need to generate any code for them. We observe
that the removed functions usually come from imported modules or are inlined
everywhere.

\subsubsection{The order of optimizations}

The ordering of optimizations is very important. Known value optimization leaves
some variables and captured variables unused, so they can be removed by
dead value elimination. It also increases the number of combinators, so we can
then inline more functions. The unused definitions left after inlining can be
pruned by dead code elimination.

The optimization level can be adjusted from the command line. Level 0 disables
all optimizations. Level 1 runs all phases in the order described above except
inlining, which is enabled from level 2. On level 3, we follow inlining with
another round of known value optimization and dead value elimination.

\section{Slot allocation}

Programs in Grit generally use many variables. It would be wasteful to allocate
a physical location for each variable, because their lifetimes are usually short
and the space would be unused most of the time. We allocate more variables to a
single location, but we must be careful not to read from a variable whose value has
been overwritten by a write to a variable assigned to the same location.

This phase corresponds to register allocation in practical compilers, but our
code generator is greatly simplified and allocates all variables to the stack
slots. We use the interference graph coloring approach
\cite{chaitin1981register,chaitin1982register,briggs1994improvements}, but the
number of colors is not limited, we only seek to minimize it. The algorithm
builds the interference graph and then greedily colors the nodes, ordered by the
decreasing number of outgoing edges.

\section{Assembler}

As the last step of the translation pipeline, we generate assembler for
IA-32 architecture from Grit. The emitted code contains all functions,
statically allocated objects and strings. To support tail-calls, the calling
convention used in Spiral is different to the C calling convention.
Arguments for C functions are placed in the caller's frame on the stack, but
during a tail-call, the frame of the caller must be discarded, so the callee
must receive the arguments in its frame.

The stack of a function with \texttt{N} slots is laid out like this (the
addresses are relative to the register \texttt{\%esp}):

\begin{ttcode}
  4*N+4  :  return address
    4*N  :  slot 0 (argument 0)
  4*N-4  :  slot 1 (argument 1)
            ...
      8  :  slot (N-2)
      4  :  slot (N-1)
      0  :  closure value
\end{ttcode}

Callee receives arguments in the first slots, placed right under the return
address (saved by the \texttt{call} instruction). Remaining slots are placed
below. The value of the function (its closure) is saved in \texttt{\%ecx},
number of arguments during unknown calls is placed in \texttt{\%eax}. The
function writes the value of \texttt{\%ecx} to the end of the stack frame, to
help the garbage collector traverse the stack. The return value is passed back
to the caller in \texttt{\%eax}.

To call a function, the caller places the arguments under its stack frame and saves
the return address by \texttt{call} instruction. A tail call overwrites the
slots of the caller, shifts the stack upward and jumps to the callee
(\texttt{jmp} instruction). Upon returning with \texttt{ret} instruction, the
callee correctly jumps to the original caller.

The code generator can use fixed registers for temporary storage, because no
variables are placed in registers. \texttt{\%eax} and \texttt{\%edx} are used
for moving the values to and from memory, \texttt{\%ecx} stores the current
function value, which is used to access the captured variables. In certain
corner cases, the register \texttt{\%ebx} is used, too. The functions also
pass around the register \texttt{\%edi} with a pointer to the runtime background
(\texttt{Bg*}).
