\chapter{Conclusion}

We presented a working implementation of a non-trivial programming language,
including a runtime and a standard library. We support first-class functions,
tail-calls and simple module system.

Our original goal was to construct a compiler for a language
Plch\footnote{\url{https://github.com/honzasp/plch}}, that was supposed to be
very similar to Spiral. The first intermediate language $^p\lambda_\chi$ was the
direct predecessor of Spine, but it featured a simple type system that allowed
to directly work with integers and functions. Grit evolved from the second
intermediate language \texttt{tac} (\texttt{t}hree \texttt{a}ddress
\texttt{c}ode), that was translated to assembler using a real register
allocator, so the quality of the generated code was better. However, we did not
manage to finish this compiler.

The original idea for the translation pipeline was to optimize in the
,,type-safe'' Spine and use Grit only as a portable code close to the assembler.
However, we found optimizing the structured Spine much harder than the simple
Grit. We also could not find a way to express some important aspects (known and
unknown calls) in a safe untyped language.

Even though Spine is in some aspects superior to certain popular languages (for
example the ,,programming language'' PHP), it has inherent drawbacks:

\begin{itemize}
  \item We are convinced that the reason why dynamically typed languages reached
    their current popularity was chiefly the lack of a high-level practical
    language with rich type system and safe memory management. Spiral, as a
    dynamic language, is thus of no value to the development of programming
    languages, because it is sitting on the same overcrowded chair as Python,
    Lua, JavaScript and all other descendants of Lisp.

  \item The language lacks any support for structures (records, objects,
    dictionaries, tables, ...), that is, tuples with symbolic names. This kind
    of objects either requires static typing (\texttt{struct} in C) or a hash
    table (objects in JavaScript, Ruby or Python). We believe that such an
    elementary operation should be cheaper than an access to a hash table.
\end{itemize}

The implementation also suffers from other problems that would have to be solved
for a real-world use:

\begin{itemize}
  \item The compiled code is quite slow, because the code generator cannot place
    variables to registers and calls an external function even for trivial
    operators such as integer addition. We also have to allocate memory for
    every single real number. Because of space and time limitations, we did not
    perform any benchmarks comparing Spiral and other languages, but we expect
    that the results would be very poor. The low-quality garbage collector
    that has to copy the whole heap during a collection, and the naive (and
    slow) algorithms in the compiler also have a negative impact.

  \item We store no source locations during parsing, so the implementation
    cannot report any hint where a reported error happened. This renders
    development of any program bigger than a few lines nearly impossible.

  \item The code generator is not only exceedingly ineffective, but also quite
    obscure and badly written.

  \item The standard libary is too minimalistic for a real use. For example, it
    lacks functions for manipulating the file system or basic data structures as
    heaps or hash tables.

  \item The language has no support for variadic arguments, so the standard
    library has to export families of functions such as \texttt{tuple-0} to
    \texttt{tuple-8} or \texttt{str-cat-0} to \texttt{str-cat-8}.
    
  \item There is not a single comment in the source code. Spiral does not even
    have a dedicated syntax for comments.
\end{itemize}

We conclude that we reached the goal that we had set for ourselves, but we can
show no real contribution of this paper. We used only the most common and
well-described algorithms, usually very simplified.
