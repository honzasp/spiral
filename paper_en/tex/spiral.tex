\chapter{Description of the language Spiral}

The language itself is very simple, as it provides only modules, functions,
variables, basic branching and literals. Primitive operations (such as
addition or array access) are defined in the standard library and implemented as
calls to the runtime. Note that the identifiers like \texttt{+} or \texttt{*}
are not built-in operators but standard variables.

No assignment operator is needed because all variables are immutable. Some
objects from the standard library, for example arrays, are internally mutable.
As a convention, functions that mutate objects have are suffixed with an
exclamation mark (\texttt{array-set!}).

Tail-calls allow an unbound number of recursive calls without the danger of
stack overflow, so all iterations can be expressed as recursion. Functions are
first-class values as in other functional languages, so there is no limitation
on storing them to variables or passing them as arguments to other functions. A
function can refer to variables defined outside of the body of the function.
We store the functions as closures consisting of the address and the values of
captured variables.

\section{Grammar}

The syntax of Spiral is based on the syntax of Scheme \cite{scheme7}. Program is
written in the fully parenthesised prefix form of s-expressions that are simple
to parse and easy to write.

\subsection{Programs and modules}
\begin{ttcode}
<top-level>   = <program> | <module>
<program>     = (program <stmt>...)
<module>      = (module <ident> <decl>...)
<decl>        = (export <ident>...) | <stmt>
\end{ttcode}

A program consists of a sequence of statements that are executed in order. A
module can also contain exports denoting the names that the module provides to
other modules and programs.

~
\subsection{Statements}

Statements can define imported names, named functions or plain variables.

\begin{ttcode}
<stmt>        = (import <import-def>...)
              | (fun <ident> (<ident>...) <stmt>...)
              | (var <ident> <expr>)
              | <expr>
<import-def>  = <ident>
              | (only <import-def> <ident>...)
              | (except <import-def> <ident>...)
              | (prefix <import-def> <ident>)
\end{ttcode}

\begin{description}

\item[\texttt{(import <import-def>...)}] The import statement loads a module and
  gives access to its exported variables in the active context. Only selected
  names can be imported using \texttt{(only ...)}, names can be excluded using
  \texttt{(except ...)} or prefixed with an identifier using \texttt{(prefix
  ...)}.

  Examples of imports:

  \begin{itemize}
    \item \texttt{(import std std.math)} imports all names from modules
      \texttt{std} and \texttt{std.math}.
    \item \texttt{(import (only std + - *))} imports from \texttt{std} just the
      names \texttt{+}, \texttt{-} and \texttt{*}.
    \item \texttt{(import (except std.math sin cos))} imports
      from~\texttt{std.math} all names except \texttt{sin} and \texttt{cos}.
    \item \texttt{(import (prefix (only std.math div mod) m.))} imports from
      \texttt{std.math} names \texttt{div} and \texttt{mod} prefixed
      \texttt{m.} (so we can refer to them as \texttt{m.div} and \texttt{m.mod}).
  \end{itemize}

\item[\texttt{(fun <fun-name> (<arg>...) <body-stmt>...)}] defines a named
  function. Multiple named functions defined next to each other can be mutually
  recursive, because a function defined earlier can refer to the functions
  defined later, including itself.

\item[\texttt{(var <var-name> <value>)}] defines a variable. Variables cannot
  refer to themselves, because their values can be used only after they are
  defined.

\item[\texttt{<expr>}] -- an expression in place of a statement is evaluated. If
  it is the last statement in a sequence of statements, its value is used as the
  value of the whole sequence, otherwise it is ignored.

\end{description}

\subsection{Expressions}

All remaining constructs of the language are expressions returning a value. If
there is no reasonable value for an expression (e.g. an empty expression
\texttt{(begin)}), it evaluates to \texttt{false}.

\begin{gram}
<expr>        = (if <expr> <expr> <expr>)
              | (cond (<expr> <stmt>...)...)
              | (when <expr> <stmt>...)
              | (unless <expr> <stmt>...)
              | (do ((<ident> <expr> <expr>)...) (<expr> <stmt>...) <stmt>...)
              | (and <expr>...)
              | (or <expr>...)
              | (begin <stmt>...)
              | (let ((<ident> <expr>)...) <stmt>...)
              | (lambda (<ident>...) <stmt>...)
              | (<expr> <expr>...)
              | (extern <ident> <expr>...)
              | <integer-literal>
              | <float-literal>
              | <string-literal>
              | <character-literal>
\end{gram}

\begin{description}
  \item[\texttt{<integer-literal>}, \texttt{<float-literal>}] are numeric
    constants that evaluate to the number they denote. Digits can be separated
    by underscores and float literals can have an exponent.

  \item[\texttt{<character-literal>}] is a character literal in single quotes
    that evaluates to the integer value of the character.

  \item[\texttt{<string-literal>}] is a string literal in double quotes. Escape
    sequences from C are supported (for example \texttt{\textbackslash{}n} is
    newline, \texttt{\textbackslash{}"} is a quote).

  \item[\texttt{(if <condition> <then> <else>)}] is a conditional expression.
    The condition is evaluated and if it is true, the first branch is evaluated,
    otherwise the second branch is evaluated. All values except \texttt{false}
    are considered true.

  \item[\texttt{(when <condition> <body-stmt>...)}] evaluates the statements in
    the body only if the condition evaluates to true.

  \item[\texttt{(unless <condition> <body-stmt>...)}] is the counterpart of
    \texttt{when} that evaluates the body if the condition is false.
    
  \item[\texttt{(cond (<condition> <stmt>...)...)}] evaluates the statements of
    the first condition that evaluates to true.

  \item[\texttt{(and <expr>...)}] evaluates expressions from left to right and
    returns the value of the first expression that evaluates to false, or
    \texttt{true} if all were true.

  \item[\texttt{(or <expr>...)}] evaluates expression from left to right and
    returns the value of the first that evaluated to true.

  \item[\texttt{(let ((<var> <expr>)...) <body-stmt>...)}] evaluates the
    expressions, bounds the values to the variables and then evaluates the
    statements in the body.

  \item[\texttt{(lambda (<arg>...) <body-stmt>...)}] creates an anonymous
    function with the given arguments. The statements in the body can access
    variables defined outside the function.

  \item[\texttt{(<fun> <arg>...)}] denotes function call. All arguments are
    evaluated from left to right, then the function is evaluated and called. It
    is an error if the first expression does not evaluate to a function.

  \item[\texttt{(extern <fun-name> <arg>...)}] is an extern function call.
    Extern functions are defined in C and are mostly used in the standard
    library. The arguments and the existence of the function cannot be checked,
    so the language does not guarantee safety.

  \item[\texttt{(begin <body-stmt>...)}] evaluates all statements and returns
    the value of the last.

  \item[\texttt{(do ((<var> <init> <next>)...) (<exit-condition> <exit-stmt>...)
    <body-stmt>...)}] is a loop expression. At the beginning, initial values of
    the variables are evaluated and bound. Then, if the condition
    (\texttt{<exit-condition>}) evaluates to true, the loop ends with evaluating
    exit statements (\texttt{<exit-stmt>...}). Otherwise, the body statements
    (\texttt{<body-stmt>...}) are evaluated, then the variables are bound to the
    values of \texttt{<next>} and the condition is checked again.

    For example, the following program computes and prints the first hundred
    Fibonacci numbers and then finishes with \texttt{done}:

\begin{spiral}
(program
  (import std)
  (do ((f1 0 f1)
       (f2 1 (+ f1 f2))
       (i  1 (+ i 1)))
    ((>= i 100)
      (println "done"))
    (println f1)))
\end{spiral}
\end{description}

\section{Standard library and basic types}

Standard library defines basic functions. Programs in Spiral have no names
defined by default, so the import of the standard library is virtually
necessary.

\begin{description}
  \item[\texttt{std.core}] defines core numeric operations (\texttt{+},
    \texttt{mod}, \texttt{<}, \texttt{==}, ...), equivalence predicates
    (\texttt{eqv?}, \texttt{equal?}) and the basic output function
    (\texttt{println}). It also defines \texttt{true} a \texttt{false} (as the
    values of \texttt{(and)} a \texttt{(or)}).

  \item[\texttt{std.array}] provides functions for working with arrays
    (\texttt{array-new}, \texttt{array-push!}, \texttt{array-get},
    \texttt{array-empty?}, ...). Arrays are indexed by integers starting from
    zero and their size and contents are mutable.

  \item[\texttt{std.tuple}] gives access to tuples with size from 0 to 8
    elements. The module defines the constructors (\texttt{tuple-2}, ...),
    access functions (\texttt{get-0}, \texttt{get-2}) and predicates
    (\texttt{tuple?}, \texttt{tuple-2?}, \texttt{tuple-n?}, ...). Tuple elements
    are immutable.

  \item[\texttt{std.cons}] defines cons pairs, used mostly to encode linked
    lists, where the first field (\emph{car}) contains the head of the list and
    the second field (\emph{cdr}) the tail. The end of the list is denoted by a
    special value (\emph{null} or \emph{nil}), in Spiral \texttt{false} is used.
    This module exports the essential functions for working with pairs
    (\texttt{cons}, \texttt{car}, \texttt{cdr}, \texttt{cons?}, ...) and linked
    lists (\texttt{list?}, \texttt{list-len}, \texttt{list-append},
    \texttt{list-reverse}, ...). Cons pairs are immutable, so it is not possible
    to create a circular list.

  \item[\texttt{std.string}] exports functions manipulating strings
    (\texttt{str-len}, \texttt{str-get}, \texttt{stringify}, \texttt{str-cat-2},
    \texttt{str-cat-3}, ...). Strings are immutable sequences of bytes.

  \item[\texttt{std.io}] provides input/output for files and standard streams
    (\texttt{io-file-open}, \texttt{io-close}, \texttt{io-stdin},
    \texttt{io-write-line}, \texttt{io-read-line}, \texttt{io-read-number},
    ...).

  \item[\texttt{std.env}] allows the program to access command line arguments
    (\texttt{env-get-argv}) and environment variables (\texttt{env-get-var}).

  \item[\texttt{std.math}] implements mathematical functions such as
    \texttt{abs}, \texttt{neg}, \texttt{sin} or \texttt{atan-2}.

  \item[\texttt{std.test}] is a minimal unit testing library.
\end{description}

Module \texttt{std} conveniently reexports the most used definitions (such as
\texttt{+} or \texttt{car}).

\section{Command line interface of the compiler}

The compiler is controlled by command line arguments:

\begin{description}
  \item[\texttt{-o}, \texttt{-{}-output}] sets the output file. By default, the
    output file is placed into the same directory and with the same basename as
    the input file. The suffix is determined by the type of output..
  \item[\texttt{-I}, \texttt{-{}-include}] adds a directory to the list of paths
    where the compiler looks for \texttt{import}ed modules.
  \item[\texttt{-e}, \texttt{-{}-emit}] controls the type of output:
    \texttt{sexpr} reads the input s-expression and pretty-prints that,
    \texttt{spiral} dumps the internal syntax tree of Spiral, \texttt{spine} and
    \texttt{grit} print the program in the respective intermediate language as
    an s-expression, \texttt{asm} dumps the internal assembler representation,
    \texttt{gas} produces input file for GNU Assembler and \texttt{exec} invokes
    the assembler, compiles the program and links it with the runtime library
    (default)..
  \item[\texttt{-t}, \texttt{-{}-runtime}] sets path to the compiled runtime
    library.
  \item[\texttt{-{}-link-cmd}] sets the linker program. The default is
    \texttt{clang} to ensure that the system linker is correctly set up to link
    to C standard library (\texttt{libc}).
  \item[\texttt{-{}-gas-cmd}] changes the command for assembler (default is
    \texttt{as}).
  \item[\texttt{-O}, \texttt{-{}-opt-level}] sets the optimization level as
    an integer from 0 to 3.
\end{description}

\section{Examples}

We conclude the chapter with a few tiny programs written in Spiral.

\subsection{Prime number generation}

The following program prints the first thousand primes. Function
\texttt{make-prime-gen} creates a generator function that returns consecutive
primes upon subsequent invocations. The function stores the found primes in an
internal array.

\begin{spiral}
(program
  (import std)
  (import std.array)
  (import std.test)

  (fun make-prime-gen ()
    (var primes (array-new))
    (lambda ()
      (fun find-next-prime (x)
        (fun check-prime (i)
          (var p (array-get primes i))
          (cond
            ((> (* p p) x) true)
            ((== (mod x p) 0) false)
            (true (check-prime (+ 1 i)))))
        (if (check-prime 1) x (find-next-prime (+ 2 x))))
      (cond
        ((== (array-len primes) 0)
          (array-push! primes 2) 2)
        ((== (array-len primes) 1)
          (array-push! primes 3) 3)
        (true 
          (var next-prime (find-next-prime (+ 2 (array-last primes))))
          (array-push! primes next-prime)
          next-prime))))

  (var gen (make-prime-gen))
  (do ((i 0 (+ i 1)))
    ((>= i 1000))
    (println (gen))))
\end{spiral}

\subsection{Computing integer powers and roots}

This program approximates $\sqrt[n]{x}$ using Newton's method (function
\texttt{root}) and computes $x^n$ by binary exponentiation (function
\texttt{power}). Both algorithms are then tested using \texttt{std.test}.

\begin{spiral}
(program
  (import std)
  (import std.math)
  (import std.test)

  (var min-del 0.000001)
  (fun root (n a)
    (fun iter (x)
      (var b (/ a (power (- n 1) x)))
      (var x-del (/ (- b x) n))
      (if (< (abs x-del) min-del) x (iter (+ x x-del))))
    (iter (* a 1.0)))
  (fun power (n a)
    (cond
      ((== n 0) 1)
      ((== n 1) a)
      ((== (mod n 2) 0) (square (power (div n 2) a)))
      (true (* a (square (power (div n 2) a))))))
  (fun square (a) (* a a))

  (var eps 0.00001)
  (test "powers" (lambda (t) 
    (assert-near-eq t eps (power 2 10) 100)
    (assert-near-eq t eps (power 3 2) 8)
    (assert-near-eq t eps (power 1 33) 33)))
  (test "roots" (lambda (t)
    (assert-near-eq t eps (root 3 1000) 10)
    (assert-near-eq t eps (root 2 49) 7)
    (assert-near-eq t eps (root 4 256) 4)))
  (test "powered roots" (lambda (t)
    (assert-near-eq t eps (power 2 (root 2 2)) 2)
    (assert-near-eq t eps (power 3 (root 3 100)) 100)
    (assert-near-eq t eps (power 2 (root 2 13)) 13)
    (assert-near-eq t 0.01 (power 5 (root 5 12345)) 12345))))
\end{spiral}
