\chapter{Runtime library}

Compiled programs need the runtime library to work properly. This library
defines the core external functions referenced by standard library and also
provides automatic memory management.

It is not a good programming practice to use mutable global variables, so we
store all ,,global'' data in an object called background (shortened to ,,bg'').
Pointer to this object is passed in register \texttt{\%edi} through the code in
Spiral and most functions in the standard library receive it as their first
argument.

\section{Value representation}

All values are represented as 32-bit numbers. Integers are stored directly,
multiplied by two to ensure that their least important bit is zero. In this way,
we can store $2^{31}$ numbers, so the range of integers in Spiral is from
$-2^{30}$ to $2^{30}-1$. Other values are stored as pointers to memory and we
distinguish them from integers by setting their least important bit to one. To
quickly recognize functions, we set the two last bits of data
objects to 01 and of functions to 11 (in binary). The original pointer can be
obtained by a binary and.

\begin{ttcode}
   <int>0 ... integer
  <ptr>01 ... data object
  <ptr>11 ... function
\end{ttcode}

All objects in memory start with a four byte pointer to static object table
(,,otable''). This table allows us to dynamically recognize the objects and
carry out some generic operations on them. It stores pointers to functions that
are used for memory management, conversion to string and equality predicates.
These functions are used to implement standard library functions
\texttt{stringify}, \texttt{eqv?} and \texttt{equal?}, because they must be able
to work with all objects.

\begin{cplusplus}
  struct ObjTable {
    const char* type_name;
    void (*stringify_fun)(Bg* bg, Buffer* buf, void* obj_ptr);
    auto (*length_fun)(void* obj_ptr) -> uint32_t;
    auto (*evacuate_fun)(GcCtx* gc_ctx, void* obj_ptr) -> Val;
    void (*scavenge_fun)(GcCtx* gc_ctx, void* obj_ptr);
    void (*drop_fun)(Bg* bg, void* obj_ptr);
    auto (*eqv_fun)(Bg* bg, void* l_ptr, void* r_ptr) -> bool;
    auto (*equal_fun)(Bg* bg, void* l_ptr, void* r_ptr) -> bool;
  };
\end{cplusplus}

The layout of memory after the pointer to the object table is dependent on the
type of the value. For example, strings store here their length and a pointer to the
character data and tuples place here their elements. Function pointers of
function objects are saved after the pointer to the object table, followed by
the captured variables. Preceding the
body of the function we store function table (,,ftable'') that contains
important information about the function, such as the number of arguments, slots
and captured variables, that are needed to correctly read the stack during
garbage collection.

\begin{cplusplus}
  struct FunTable {
    uint32_t slot_count;
    uint32_t arg_count;
    uint32_t capture_count;
    const char* fun_name;
    uint8_t padding_0[16];
  };
\end{cplusplus}

\section{Memory management}

The object memory is separated into chunks forming a linked list. When the
program requests memory to allocate an object, the runtime library uses the
free memory of the first chunk. If there is not enough space, the runtime
library asks the system for another chunk and places it at the beginning of the
list. However, if the program has allocated too much memory, the garbage
collector is invoked and copies all live objects to fresh memory
locations and then releases old chunks.

All objects referenced by variables and all objects referenced by live objects
are considered live. We first evacuate (copy from the old memory space to the
new and save from destruction) all objects found on the stack. Then we scavenge
the evacuated objects and evacuate all referenced objects. To prevent an object
from being evacuated and copied multiple times, we rewrite it with a forward
pointer pointing to the new location after evacuation. The remaining objects
that were not evacuated are then dropped.

Some objects, such as arrays and strings, store their contents in an additional
memory space independent from the object memory. These objects are responsible
for their memory management and release the memory when they are dropped.

\section{Interface between the programs and the library}

Call to an external function in Spiral program invokes a C function of the
same name. Passed arguments are the pointer to the background (from register
\texttt{\%edi}) and the top of the stack (register \texttt{\%esp}), followed by
the arguments from the Spiral call. Because the runtime is written in
\Cplusplus{}, the functions that are called from Spiral are placed inside
\texttt{extern "C"~\{ ... \}} blocks, because the \Cplusplus{} compiler would
otherwise mangle their names. Some variables, for example objects \texttt{true}
and \texttt{false}, would be inconvenient to define in this way, so we use their
mangled names in compiled Spiral programs.
